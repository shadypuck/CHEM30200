\documentclass[../notes.tex]{subfiles}

\pagestyle{main}
\renewcommand{\chaptermark}[1]{\markboth{\chaptername\ \thechapter\ (#1)}{}}
\setcounter{chapter}{8}

\begin{document}




\chapter{???}
\section{NMR Spectroscopy}
\begin{itemize}
    \item \marginnote{2/28:}Announcements.
    \begin{itemize}
        \item Second HW is due Friday at noon.
        \begin{itemize}
            \item We should have everything we need for it after today.
        \end{itemize}
        \item The final will be from 8:00-9:20am on Tuesday.
        \begin{itemize}
            \item It is solely on Anderson's lectures (it's just a second midterm).
            \item Not open note. Shouldn't be too many things to memorize: $g\beta H$, magnetic moment formulas, gnarly stuff like the EXAFS formula we don't need to have memorized.
        \end{itemize}
    \end{itemize}
    \item We now start the lecture.
    \item Relation between EPR and NMR.
    \begin{itemize}
        \item There are many parallels.
        \item NMR is more common than EPR, and much more complicated.
    \end{itemize}
    \item We will start with some simpler NMR examples to build a foundation.
    \item NMR background/underlying principles.
    \begin{itemize}
        \item Like an electron has $S=\pm 1/2$, nuclei also have a given spin $I$.
        \item While a single electron can only be $S=1/2$, a single nucleus can be $I=0,1/2,1,3/2,\dots$.
    \end{itemize}
    \item A few rules about $I$.
    \begin{enumerate}
        \item A nucleus with an odd mass number has a half-integer spin.
        \item A nucleus with an even mass number and an odd atomic number has an integer spin.
        \begin{itemize}
            \item Examples: Deuterium, \ce{{}^14N}, etc.
        \end{itemize}
        \item A nucleus with an even mass number and an even atomic number has zero spin.
    \end{enumerate}
    \item We now investigate the \textbf{Zeeman splitting} of an $I=1/2$ nucleus.
    \begin{itemize}
        \item Like electron spins in EPR, nuclear spins align with or against applied magnetic fields.
        \item The splitting is as in Figure \ref{fig:zeroFieldSplita}.
        \item We know that $\Delta E=\gamma\hbar B_0$, where $\gamma$ is the gyromagnetic ratio and $B_0$ is the applied field.
    \end{itemize}
    \item The notes also consider the $I=1$ case.
    \item \textbf{Zeeman splitting}: The energy difference between a spin aligned with and against a magnetic field.
    \item \textbf{Larmor frequency}: The frequency of electromagnetic radiation that induces a spin flip. \emph{Also known as} \textbf{resonant frequency}. \emph{Denoted by} $\bm{\nu_0}$. \emph{Given by}
    \begin{equation*}
        \nu_0 = \frac{\gamma B_0}{2\pi}
    \end{equation*}
    \begin{itemize}
        \item Takeaway: A nucleus's resonant frequency is determined by $\gamma$ and $B_0$.
    \end{itemize}
    \item Typical values for $B_0$, $\gamma$, and $\nu_0$.
    \begin{itemize}
        \item $B_0$: \SIrange{1.4}{14}{\tesla}.
        \item $\nu_0$: \SIrange{60}{600}{\mega\hertz}, though we can go higher.
    \end{itemize}
    \item Example: The Larmor frequency of some common NMR nuclei under a \SI{9.4}{\tesla} magnet.
    \begin{itemize}
        \item \ce{{}^1H}: We have that
        \begin{equation*}
            \nu_0 = \frac{\gamma B_0}{2\pi}
            = \frac{(\SI{26.8e7}{\per\tesla\per\second})(\SI{9.4}{\tesla})}{2\pi}
            = \SI{4.0e8}{\per\second}
            = \SI{400}{\mega\hertz}
        \end{equation*}
        \item \ce{{}^13C}: \SI{100}{\mega\hertz}.
        \item \ce{{}^31P}: \SI{162}{\mega\hertz}.
        \item \ce{{}^2H}: \SI{61}{\mega\hertz}.
    \end{itemize}
    \item The key limitation of NMR is \emph{sensitivity}.
    \begin{itemize}
        \item The sensitivity for NMR is pretty atrocious. This is because all relevant energies are pretty small (radiofrequency region) and thus hard to detect.
        \item Quantitatively, sensitivity is proportional to the magnetogyric ratio cubed times the number of nuclei (equivalently, the concentration).
        \item This is why MRI is really tough.
        \item Also why you need really big magnets for NMR.
        \item Additional limitation: The signal increases by the square of the applied field strength.
        \begin{itemize}
            \item How is this limiting??
        \end{itemize}
    \end{itemize}
    \item Aside: Hyperpolarization for NMR.
    \begin{itemize}
        \item Goal: Enhance signal intensity by further polarizing nuclear spins.
        \item Essentially, if you have four particles in the spin ground state and three in the spin excited state, exciting one of the ground state spins won't cause a significantly measurable change. However, if you \emph{hyperpolarize} the system so that there are, for example, six particles in the ground state and only one in the excited state, then inducing an excitation causes a far greater change.
        \begin{itemize}
            \item You can do this with \textbf{dynamic nuclear polarization}, particularly \textbf{triplet DNP}.
            \item Also something akin to ENDOR with Zeeman splitting and microwaves to get a huge polarization buildup. Nobody has pulled this off \emph{in situ} yet, but that's a goal.
        \end{itemize}
        \item There's a scientist in Texas (possibly Christian Hilty??) looking into \textbf{parahydrogen}.
        % \item DNP: Taking a singlet. Photoexciting it with light to generate a triplet. Then intersystem crossing to give you an $M_s$ that is selectively 0. You selectively populate the $M_s=0$ state. If this interacts with your nuclear spin, it causes the polarization; this is \textbf{triplet dynamic nuclear polarization}.
    \end{itemize}
    \item \textbf{Dynamic nuclear polarization}: A technique for hyperpolarization involving the transfer of spin polarization from electrons to nuclei. \emph{Also known as} \textbf{DNP}.
    \item \textbf{Triplet DNP}: Photoexciting a singlet to a triplet and then harnessing intersystem crossing to selectively populate the $M_s=0$ state before transferring this polarization to the nuclei.
    \item \textbf{Parahydrogen}: The spin isomer of hydrogen with the two proton spins aligned antiparallel.
    \item The key advantage of NMR is \emph{resolution}.
    \begin{itemize}
        \item The specific frequency of a given nuclei is often exquisitely sensitive to its chemical environment.
        \item NMR is a Qbit technique: We're using the nuclear spin as a quantum sensor for the system.
        \item This really comes down to shielding.
    \end{itemize}
    \item Shielding.
    \begin{itemize}
        \item A useful (but technically inaccurate) classical analogy.
        \begin{itemize}
            \item If we have an electron and we apply a magnetic field $H_0$ to it in the $z$-direction, our electron will begin to circulate around it and induce a magnetic field in the opposite ($-z$) direction.
            \item The angular frequency $\omega_1$ equals $eH_0/2m_e$.
        \end{itemize}
        \item Takeaway: Applied magnetic fields result in slight changes depending on the nuclear environment.
        \item Because nuclei have very small energy splittings, we can resolve very small energy changes in our NMR spectra; this is the origin of the ppm splitting we're familiar with.
    \end{itemize}
    \item Shielding originates primarily from three places.
    \begin{enumerate}
        \item Nucleus: Specifically how electron-rich or -poor it is.
        \item Solvent: The electrons therein will strongly influence the magnetic field felt by the nuclei.
        \item Chemical environment: Bonds, functional groups, other nuclei, electrons, etc.
    \end{enumerate}
    \item Parts 1 and 3 (electron richness and nuclear factors) are related to each other and are important.
    \begin{itemize}
        \item We usually correct for part 2 using a known solvent shift and an internal standard.
    \end{itemize}
    \item Measuring and describing the \textbf{chemical shift}.
    \begin{itemize}
        \item \SI{1}{\ppm} is a shift in the resonant frequency of a given nucleus. Specifically,
        \begin{equation*}
            \SI{1}{\ppm} = \frac{\nu_1-\nu_\text{ref}}{\nu_0\cdot\num{e6}}
        \end{equation*}
        \begin{itemize}
            \item $\nu_1$ is the measured Larmor frequency for a given nucleus.
            \item $\nu_\text{ref}$ refers to a single compound that we reference against, e.g., TMS.
            \item $\nu_0$ is the predicted Larmor frequency for said nucleus based on its $\gamma$.
        \end{itemize}
        \item Thus, at \SI{60}{\mega\hertz} (for example), $\SI{1}{\ppm}=\SI{60}{\hertz}$.
        \item Chemical shift lingo.
        \begin{itemize}
            \item More \emph{shielded} nuclei require \emph{higher fields}, have a \emph{lower chemical shift}, and are positioned relatively \emph{upfield}.
            \item More \emph{deshielded} nuclei resonate at \emph{lower fields}, have a \emph{higher chemical shift}, and are positioned relatively \emph{downfield}.
        \end{itemize}
    \end{itemize}
    \item Line widths.
    \begin{itemize}
        \item Recall from EPR: $\tau$ denotes \textbf{lifetime}.
        \item We're limited by the Heisenberg uncertainty principle
        \begin{equation*}
            \Delta E\Delta t = \hbar
            \quad\Longleftrightarrow\quad
            \Delta E = \frac{\hbar}{\tau}
        \end{equation*}
        \item It follows that a long lifetime $\tau$ gives you a small $\Delta E$ and a sharp line.
        \item Typical lifetimes: $\tau_\text{NMR}=\SIrange{1}{100}{\second}$, $\tau_\text{rot}=\SI{e-9}{\second}$, $\tau_\text{vib}=\SI{e-6}{\second}$.
    \end{itemize}
    \item \textbf{Lifetime}: The time that a signal will remain polarized. \emph{Denoted by} $\bm{\tau}$, $\bm{\Delta t}$.
    \item \textbf{Chemical shift}. \emph{Denoted by} $\bm{\sigma}$. \emph{Given by}
    \begin{equation*}
        \sigma = \sigma_d+\sigma_p+\sigma_R+\sigma_e+\sigma_\text{int}
    \end{equation*}
    \begin{itemize}
        \item Each of the five variables represents a contribution to the nucleus from some magnetic component.
        \item $\sigma_d$ is our diamagnetic term.
        \begin{itemize}
            \item Usually positive, so it causes signals to shift downfield.
            \item Most important for \ce{{}^1H} NMR.
            \item Correlates with the $s$-electron density at the nucleus; thus can give us a direct readout of the electron density of the nucleus. Also, more electron rich means more shielded.
        \end{itemize}
        \item $\sigma_p$ is our paramagnetic term.
        \begin{itemize}
            \item Usually negative, so it causes signals to shift upfield.
            \item Most important for heavier elements (e.g., \ce{{}^31P}, \ce{{}^119Sn}, etc.).
            \item A large term in general (larger than $\sigma_d$??).
            \item Distinct usage from paramagnetic \emph{samples} (diamagnetic nuclei can have paramagnetic terms).
        \end{itemize}
        \item $\sigma_R$ is for ring currents.
        \begin{itemize}
            \item Consider a benzene ring for instance. The large induced ring current creates two cones of positive shift and negative regions outside.
            \item Other multiple bonds also contribute in various ways.
            \item For more details, see the discussion associated with Figure \ref{fig:BnzRingCurrent} below.
        \end{itemize}
        \item $\sigma_e$ is for electric fields.
        \item $\sigma_\text{int}$ is for intermolecular effects.
        \item The latter three are often combined into $\sigma_N$.
        \item Recall\dots
        \begin{itemize}
            \item $\sigma>0$ means an upfield shift (lower ppm)
            \item $\sigma<0$ means a downfield shift (higher ppm).
        \end{itemize}
    \end{itemize}
    \item Benzene ring current effects.
    \begin{figure}[h!]
        \centering
        \begin{tikzpicture}
            \fill [blz,opacity=0.5] (0.45,0) arc[start angle=0,end angle=180,x radius=0.45cm,y radius=0.135cm] -- (-1,-2) arc[start angle=180,end angle=0,x radius=1cm,y radius=0.3cm] -- cycle;
            \fill [blz,opacity=0.5] (0.45,0) arc[start angle=0,end angle=-180,x radius=0.45cm,y radius=0.135cm] -- (-1,-2) arc[start angle=-180,end angle=0,x radius=1cm,y radius=0.3cm] -- cycle;
    
            \begin{scope}[yscale=0.3,rotate=-2]
                \node[transform shape]{\chemfig{**6(------)}};
            \end{scope}
    
            \fill [blz,opacity=0.5] (0.45,0) arc[start angle=0,end angle=180,x radius=0.45cm,y radius=0.135cm] -- (-1,2) arc[start angle=180,end angle=0,x radius=1cm,y radius=0.3cm] -- cycle;
            \fill [blz,opacity=0.5] (0.45,0) arc[start angle=0,end angle=-180,x radius=0.45cm,y radius=0.135cm] -- (-1,2) arc[start angle=-180,end angle=0,x radius=1cm,y radius=0.3cm] -- cycle;
    
            \footnotesize
            \node at (0,2) {$+$};
            \node at (0,-2) {$+$};
            \node at (1.3,0) {$-$};
            \node at (-1.3,0) {$-$};
        \end{tikzpicture}
        \caption{The effect of the ring current in benzene.}
        \label{fig:BnzRingCurrent}
    \end{figure}
    \begin{itemize}
        \item It's a fake ring current.
        \item It's a quantum mechanical effect from the interaction of the magnetic field with the degenerate $\pi$ system.
        \item But still, it's a useful classical picture.
        \item Something with Zwitterionic character may be able to give you a "real" ring current.
    \end{itemize}
    \item Most of us have probably seen some of this before. But now we'll move onto something we haven't seen: Heavier nuclei.
    \item With heavier nuclei, $\sigma_p$ dominates.
    \begin{itemize}
        \item The spread is much bigger than with \ce{{}^1H} NMR.
        \item $\sigma_d$ is still there but is typically small and much more constant.
    \end{itemize}
    \item Contributions to $\sigma_p$.
    \begin{itemize}
        \item The orbital angular momentum, largely from the $p,d,f$ orbitals.
        \begin{itemize}
            \item We do \emph{not} consider electron spin here, just orbital spin.
        \end{itemize}
        \item The magnitude of $\sigma_p$ is controlled by mixing.
        \begin{itemize}
            \item Mixing in other states mixes in their angular momentum.
            \item Most common when there is lowered symmetry, low-lying excited states, and lots of EWGs.
            \item Mathematically, we describe $\sigma_p$ with the \textbf{Ramsey equation}.
        \end{itemize}
    \end{itemize}
    \item \textbf{Ramsey equation}: The equation giving the relative magnitude of $\sigma_p$. \emph{Given by}
    \begin{equation*}
        \sigma_p \propto -\left[ \frac{1}{E_\text{es}}-E_\text{gs} \right]\left\langle \frac{1}{r^3} \right\rangle[\pi\text{-bonding term}]
    \end{equation*}
    \begin{itemize}
        \item es denotes the excited state.
        \item gs denotes the ground state.
        \item Rationalizing the energy separation (first) term.
        \begin{itemize}
            \item Alkanes have strong bonds and no low-lying excited states, while alkenes will have lower excited states.
            \item Thus, $\sigma_p$ is greater for alkenes than alkanes, for example.
            \item Takeaway: Unsaturation typically leads to larger paramagnetic terms.
        \end{itemize}
        \item Rationalizing the $1/r^3$ term.
        \begin{itemize}
            \item It is related to how close electrons are to the nucleus.
            \item EWGs decrease electron-electron repulsions, moving electrons closer to the nucleus, thus increasing $\sigma_p$ and shift heavier nuclei more downfield.
        \end{itemize}
        \item The $\pi$-bonding term is complicated; we will not discuss it further.
        \begin{itemize}
            \item It is summed over all relevant interactions.
            \item Depends on the system at hand.
        \end{itemize}
    \end{itemize}
    \item Thus, the chemical shift can provide important chemical information.
    \item However, it can also provide other useful information, such as on \textbf{coupling}.
    \item \textbf{First order system}: A system in which the spread of the nuclei is much higher in ppm than the coupling constant, i.e., in which the following equation is satisfied.
    \begin{equation*}
        \Delta\nu = |\nu_{\ce{A}}-\nu_{\ce{X}}| \gg J_{\ce{AX}}
    \end{equation*}
    \begin{itemize}
        \item $\nu_{\ce{A}}$ is the frequency/ppm shift of nucleus \ce{A}.
        \item $\nu_{\ce{X}}$ is the frequency/ppm shift of nucleus \ce{X}.
        \item $J_{\ce{AX}}$ is the coupling constant between \ce{A} and \ce{X}.
        \item Coupling is relatively simple in first order systems.
        \item We will not get into second order splitting, which occurs when the above inequality is not satisfied and is very complicated.
    \end{itemize}
    \item Example: Consider the following molecule, which has two nuclei of interest (each with $I=1/2$).
    \begin{figure}[H]
        \centering
        \footnotesize
        \chemfig{Cl_3C-C(-[6]Cl)(-[2]{H_A})-C(-[6]Cl)(-[2]{H_X})-Cl}
        \caption{NMR coupling of two neighboring nuclei.}
        \label{fig:couplingTwo}
    \end{figure}
    \begin{itemize}
        \item Comments on the molecule.
        \begin{itemize}
            \item A nasty molecule chemically; would probably destroy the ozone layer and such.
            \item Only two spin-active proton nuclei.
            \item We stick on the \ce{CCl3} group to put the two protons in different chemical environments (gets rid of the reflection plane that would be there if \ce{CCl3} were just \ce{Cl}).
        \end{itemize}
        \item Let's look at the peak structure of \ce{A} first.
        \begin{itemize}
            \item \ce{X} can have one of two values: $M_I=\pm 1/2$.
            \item Thus, \ce{A} can be in two marginally different chemical environments, and hence should appear as a doublet.
        \end{itemize}
        \item The same is true of \ce{X}.
        \item Thus, the full \ce{{}^1H} NMR spectrum should consist of a pair of doublets.
    \end{itemize}
    \item There are two types of coupling.
    \begin{enumerate}
        \item \textbf{Dipolar coupling}. Occurs "through space."
        \begin{figure}[h!]
            \centering
            \begin{subfigure}[b]{0.2\linewidth}
                \centering
                \begin{tikzpicture}
                    \small
                    \node (A)          {\ce{A}};
                    \node (X) at (1,0) {\ce{X}}
                        edge [blx,thick,-stealth] (1,0.7)
                    ;
                    \draw [blx,dashed,->] (1,0.7) to[out=90,in=90,in looseness=3,out looseness=1] (A);
                    \draw [blx,dashed,->] (A) to[out=-90,in=-90,looseness=3] (X);
        
                    \footnotesize
                    \draw [-latex] (-0.5,-0.5) -- node[left]{$B_0$} ++(0,1);
        
                    \path (-1,0) -- (2,0);
                \end{tikzpicture}
                \caption{Opposing.}
                \label{fig:dipolarCouplinga}
            \end{subfigure}
            \begin{subfigure}[b]{0.2\linewidth}
                \centering
                \begin{tikzpicture}
                    \small
                    \node (A)          {\ce{A}};
                    \node (X) at (1,0) {\ce{X}}
                        edge [blx,thick,stealth-] (1,0.7)
                    ;
                    \draw [blx,dashed,<-,shorten <=2pt] (1,0.7) to[out=90,in=90,in looseness=3,out looseness=1] (A);
                    \draw [blx,dashed,<-] (A) to[out=-90,in=-90,looseness=3] (X);
                \end{tikzpicture}
                \caption{Reinforcing.}
                \label{fig:dipolarCouplingb}
            \end{subfigure}
            \caption{Dipolar coupling.}
            \label{fig:dipolarCoupling}
        \end{figure}
        \begin{itemize}
            \item Having a nearby element \ce{X} that is magnetized is like bringing an additional magnet close to \ce{A}, affecting its chemical environment.
            \item \ce{X} can oppose or reinforce the applied magnetic field $B_0$ at \ce{A} (see Figure \ref{fig:dipolarCoupling}).
            \item The field is given by
            \begin{equation*}
                B_{\ce{AX}} = \gamma_{\ce{A}}\gamma_{\ce{X}}\cdot\frac{3\cos^2\theta-1}{r_{\ce{AX}}^2}
            \end{equation*}
            \begin{itemize}
                \item $\gamma$'s are gyromagnetic ratios.
                \item $r_{\ce{AX}}$ is assumed to be big; thus distance is important.
                \begin{itemize}
                    \item Is the exponent a 2 or a 3??
                \end{itemize}
                \item $\theta$ is the angle between the two nuclear spin axes (of \ce{A} and \ce{X}).
                \item What exactly is this field??
            \end{itemize}
            \item Dipolar coupling averages to zero in solution but still affects relaxation.
            \item When \ce{{}^31P} NMR is run, it is done proton decoupled to get bigger shifts.
            \begin{itemize}
                \item How is this relevant??
            \end{itemize}
        \end{itemize}
        \item \textbf{Scalar coupling}. Occurs "through bond."
        \begin{itemize}
            \item Dominates in solution.
            \item We have that
            \begin{equation*}
                E_\text{scalar} = hJM_{I,\ce{A}}M_{I,\ce{X}}
            \end{equation*}
            \item $J$ is our coupling constant in Hertz.
            \item Notes on $J$.
            \begin{itemize}
                \item Can be positive or negative, but in most cases, we don't care what its sign is. We can determine this experimentally if we need it, though.
                \item Proportional to the $s$-character of the bond.
                \item Independent of $H$, so the same on all instruments.
            \end{itemize}
        \end{itemize}
    \end{enumerate}
    \item Dipolar coupling example: The \ce{C3Cl6H2} doublets.
    \begin{figure}[h!]
        \centering
        \begin{tikzpicture}[
            every node/.style=black
        ]
            \small
            \draw (0,3) node[above]{$E$} -- (0,0) -- (6,0) node[right]{$H$};
    
            \footnotesize
            \draw [dashed]
                (1.25,2.3) -- (3,2.3)
                (1.25,0.7) -- (3,0.7)
            ;
    
            \draw [blx,ultra thick]
                (0,1.5) -- ++(0.5,0)
                (1.25,2.3) -- ++(0.5,0)
                (1.25,0.7) -- node[below]{$-\frac{\nu_{\ce{A}}}{2}$} ++(0.5,0)
                (2.5,2.6) -- ++(0.7,0)
                (2.5,2) -- ++(0.7,0)
                (2.5,1) -- ++(0.7,0)
                (2.5,0.4) -- ++(0.7,0)
            ;
            \draw [blx,semithick]
                (0.5,1.5) -- node[above left]{$\beta$} (1.25,2.3)
                (0.5,1.5) -- node[below left]{$\alpha$} (1.25,0.7)
                (1.75,2.3) -- (2.5,2.6)
                (1.75,2.3) -- (2.5,2)
                (1.75,0.7) -- (2.5,1)
                (1.75,0.7) -- (2.5,0.4)
            ;
    
            \node at (3.6,3.1) {\ce{A}};
            \node at (3.6,2.6) {$\beta$};
            \node at (3.6,2) {$\beta$};
            \node at (3.6,1) {$\alpha$};
            \node at (3.6,0.4) {$\alpha$};
    
            \node at (3.9,3.1) {\ce{X}};
            \node at (3.9,2.6) {$\beta$};
            \node at (3.9,2) {$\alpha$};
            \node at (3.9,1) {$\alpha$};
            \node at (3.9,0.4) {$\beta$};
    
            \node at (4.7,3.1) {$M_{I,\ce{A}}$};
            \node at (4.7,2.6) {$-\frac{1}{2}$};
            \node at (4.7,2) {$-\frac{1}{2}$};
            \node at (4.7,1) {$+\frac{1}{2}$};
            \node at (4.7,0.4) {$+\frac{1}{2}$};
    
            \node at (5.7,3.1) {$M_{I,\ce{X}}$};
            \node at (5.7,2.6) {$-\frac{1}{2}$};
            \node at (5.7,2) {$+\frac{1}{2}$};
            \node at (5.7,1) {$+\frac{1}{2}$};
            \node at (5.7,0.4) {$-\frac{1}{2}$};
    
            \draw [rex,thick,stealth-stealth] (2.65,2.3) -- (2.65,2.6);
            \draw [orx,thick,stealth-stealth] (2.85,2) -- (2.85,2.6);
            \draw [ylx,thick,stealth-stealth] (2.65,1) -- (2.65,2);
            \draw [grx,thick,stealth-stealth] (3.05,0.4) -- (3.05,2.6);
        
            \draw [rex,ultra thick] (6.2,2)   -- ++(0.5,0) node[right]{$\frac{J}{4}=JM_{I,\ce{A}}M_{I,\ce{X}}$};
            \draw [orx,ultra thick] (6.2,1.5) -- ++(0.5,0) node[right]{$\frac{J}{2}$};
            \draw [ylx,ultra thick] (6.2,1)   -- ++(0.5,0) node[right]{$\nu_{\ce{A}}-\frac{J}{2}$};
            \draw [grx,ultra thick] (6.2,0.5) -- ++(0.5,0) node[right]{$\nu_{\ce{A}}+\frac{J}{2}$};
    
            \path (-2,0) -- (8,0);
        \end{tikzpicture}
        \caption{Formation of a pair of NMR doublets.}
        \label{fig:NMRpairDoublets}
    \end{figure}
    \begin{itemize}
        \item Analysis of the first splitting (originates from the applied magnetic field).
        \begin{itemize}
            \item Of the left two connecting lines, the lower one is referred to as the \textbf{$\bm{\alpha}$-manifold} and the upper one is referred to as the \textbf{$\bm{\beta}$-manifold}.
            \begin{itemize}
                \item What exactly are these?? Relation to $\alpha,\beta$ designation of spin from \textcite{bib:CHEM26100Notes}.
            \end{itemize}
            \item The energy of the upper state at the end of the $\beta$-manifold is given by
            \begin{equation*}
                E = -\frac{\gamma}{2\pi}B_0(1-\sigma_{\ce{A}})M_{I,\ce{A}}
                = -\nu_{\ce{A}}M_{I,\ce{A}}
                = \frac{\nu_{\ce{A}}}{2}
            \end{equation*}
            \begin{itemize}
                \item There is a sign flip in the last equality because $M_{I,\ce{A}}$ is negative.
            \end{itemize}
            \item Per the definitions of the green and yellow gaps in Figure \ref{fig:NMRpairDoublets}, the energy difference between the two states at the end of the $\alpha$- and $\beta$-manifolds will just be $\nu_{\ce{A}}$.
            \item Takeaway: In the absence of coupling, one peak corresponds to \ce{A} at the frequency $\nu_{\ce{A}}$.
        \end{itemize}
        \item We now factor in the second splitting (originates from coupling).
        \begin{itemize}
            \item We label the resultant states with a second set of $\beta$'s and $\alpha$'s.
            \begin{itemize}
                \item The signs are related to the signs of the fields from the other nucleus.
            \end{itemize}
            % \item $JM_{I,\ce{A}}M_{I,\ce{X}}=J/4$ is the secondary splitting.
            % \begin{itemize}
            %     \item Thus, $J/2$ is the energy difference between $\beta\beta$ and $\beta\alpha$.
            % \end{itemize}
            % \item Similarly\dots
            % \begin{itemize}
            %     \item $\nu_{\ce{A}}-J/2$ is the energy difference between $\beta\alpha$ and $\alpha\alpha$.
            %     \item $\nu_{\ce{A}}+J/2$ is the energy difference between $\alpha\beta$ and $\beta\beta$.
            % \end{itemize}
            \item Selection rule: $\Delta M_I=\pm 1$.
            \begin{itemize}
                \item Thus, only $\alpha\alpha\to\beta\alpha$ (yellow) and $\alpha\beta\to\beta\beta$ (green) are allowed.
            \end{itemize}
            \item From the frequencies of these two transitions, the splitting in the doublet is
            \begin{equation*}
                J = \left( \nu_{\ce{A}}+\frac{J}{2} \right)-\left( \nu_{\ce{A}}-\frac{J}{2} \right)
            \end{equation*}
            \item $J$ is typically small compared to $\nu_{\ce{A}}$.
            \item Takeaway: Factoring in coupling, we will observe a \emph{doublet} centered around $\nu_{\ce{A}}$. However, $\nu_{\ce{A}}$ is no longer an allowable transition; only the yellow and green ones are (note that these are centered around $\nu_{\ce{A}}$). The peaks corresponding to the yellow and green transitions will be separated by $J$, as described above.
        \end{itemize}
    \end{itemize}
    \item Scalar coupling example: A diatomic molecule.
    \begin{figure}[H]
        \centering
        \begin{tikzpicture}
            \footnotesize
            \node [circle,draw=rex] (A) {A}
                edge [blx,thick,-stealth,shorten <=1mm] (0,0.8)
            ;
            \node [circle,draw=rex] (X) at (1.5,0) {X}
                edge [rex] node[above]{${\color{blx}\downharpoonleft\hspace{-1pt}\upharpoonright}$} (A)
                edge [blx,thick,stealth-,shorten <=1mm] (1.5,0.8)
            ;
        \end{tikzpicture}
        \caption{Scalar coupling in a diatomic molecule.}
        \label{fig:scalarCouplingDiatomic}
    \end{figure}
    \begin{itemize}
        \item How scalar coupling works in more detail (very complicated, but a nice simplistic picture).
        \begin{itemize}
            \item Empirical observation: Electrons and nuclei prefer to align antiparallel.
            \item Notice how each nucleus in Figure \ref{fig:scalarCouplingDiatomic} is aligned antiparallel to the nearest electron.
            \item Naturally, the electrons are also aligned antiparallel since they are in the same bonding orbital.
            \item Thus, the neighboring nuclei prefer to be anti-parallel because of the indirect interaction chain of nucleus-electron, electron-electron, electron-nucleus.
        \end{itemize}
        \item At this point, some of our previous observations should make more sense.
        \begin{itemize}
            \item Example: Scalar coupling scales with $s$-character because greater $s$-character maximizes electron-nucleus interactions.
        \end{itemize}
        % \item As \ce{A} is polarized, there will be some coupling constant to \ce{X}.
        % \item As our nuclear spins (big arrows) point one direction, they polarize the bonding electrons in the other direction.
        % \item Takeaway: Neighboring spins will be antiparallel.
        \item We have no way of measuring the sign of $J$ in an NMR experiment. In order to do the transition, we can see why $s$-character matters: The more the electrons interact with the nucleus (higher $s$-character means closer to nuclei), the stronger the effect will be.
        \begin{itemize}
            \item What is the effect?? What is scalar coupling? How does it show up? How do all of the equations fit together?
        \end{itemize}
        \item On the sign of $J$ for neighboring nuclei.
        \begin{itemize}
            \item If we have a single bond, then $J>0$ and antiparallel spins are more stable.
            \item If we have a double bond, then $J<0$ and parallel spins are more stable.
            \item In general, an odd number of bonds means $J>0$ and an even number of bonds means $J<0$.
            \begin{itemize}
                \item There are many exceptions, though.
            \end{itemize}
        \end{itemize}
        \item On the magnitude of $J$.
        \begin{itemize}
            \item 2-bond: ${}^2J_{\ce{HH}}=\SIrange{10}{25}{\hertz}$. Are these values negative?? $sp^2$ can be $+\SI{41}{\hertz}$.
            \item 3-bond: ${}^3J_{\ce{HH}}=\SIrange{0}{25}{\hertz}$.
        \end{itemize}
    \end{itemize}
    \item Coupling constants and electronic proximity to the nucleus.
    \begin{table}[h!]
        \centering
        \small
        \renewcommand{\arraystretch}{1.2}
        \begin{tabular}{ccc|ccc}
            \textbf{Compound} & \textbf{Hybridization} & \textbf{$\bm{J_\textbf{CH}}$ (Hz)} & \textbf{EWG} & $\bm{\chi}$ & \textbf{$\bm{J_\textbf{CH}}$ (Hz)}\\
            \hline
            Ethane    & $sp^3$ & 125 & \ce{CH3-F}         & 4.0  & 150\\
            Ethene    & $sp^2$ & 156 & \ce{CH3-Cl}        & 3.2  & 150\\
            Benzene   & $sp^2$ & 159 & \ce{CH3-OH}        & 3.4  & 141\\
            Cubane    &        & 160 & \ce{CH4}           & 2.2  & 125\\
            Acetylene & $sp$   & 248 & \ce{Cp2Zr(CH3)(I)} & 1.3  & 120\\
            \ce{H-H}  & $s$    & 284 & \ce{CH3-Li}        & 0.98 & 98\\
        \end{tabular}
        \caption{Coupling constants and molecular electronics.}
        \label{tab:couplingElectron}
    \end{table}
    \begin{itemize}
        \item The last coupling constant on the left is technically a $J_{\ce{HH}}$, not a $J_{\ce{CH}}$.
        \item The left exhibits the expected trend based on $s$-character: Compounds that maximize $s$-character and place electrons closer to the nucleus have higher coupling constants.
        \item The right exhibits the expected trend based on \textbf{Bent's rule}: Compounds with less electronegative EWGs must give more $s$-character to bonding; thus, there is less $s$-character to promote (scalar) coupling and the coupling constant diminishes.
        \item You can quantitatively derive this stuff with MO theory.
        \item Observing these coupling constants: They show up under \ce{{}^13C} NMR that isn't proton decoupled.
    \end{itemize}
    \item \textbf{Bent's rule}: More electronegative elements prefer $p$-character.
    \begin{itemize}
        \item See \textcite{bib:CHEM20100Notes} for more.
    \end{itemize}
    \item Aside: Diamond vacancy centers in quantum computing.
    \begin{itemize}
        \item One of the first things used for a quantum computer was an NMR.
        \item The first quantum computation was performed via NMR spectroscopy.
        \item It was a funky molecule, just enough stuff to do quantum computation, but it worked.
    \end{itemize}
    \item A couple more notes about coupling constants.
    \begin{figure}[h!]
        \centering
        \footnotesize
        \begin{subfigure}[b]{0.3\linewidth}
            \centering
            \chemfig{M(-@{P1}PR_3)(-[2]@{P2}PR_3)(-[4]R_3@{P3}P)(-[6,,,,black!20]{\color{black!20}P}|{\color{black!20}R_3})}
            \chemmove{
                \draw [rex,thick,-,shorten <=3pt,shorten >=12pt] (P1) to[out=90,in=0,looseness=1.2] node[above right,black]{$J_\textit{cis}$} (P2);
                \draw [rex,thick,-,shorten <=3pt,shorten >=3pt] (P1) to[out=-90,in=-90,looseness=1.5] node[pos=0.25,below right,black]{$J_\textit{trans}$} (P3);
            }
            \caption{\emph{cis} vs. \emph{trans}.}
            \label{fig:couplingSpecialCasea}
        \end{subfigure}
        \begin{subfigure}[b]{0.3\linewidth}
            \centering
            \chemfig{(-[:120]H)--[:60]H}
            \hspace{2em}
            \begin{tikzpicture}
                \draw circle (4mm);
    
                \draw (90:0.4) -- (90:0.6) node(H1)[above]{\ce{H}};
                \draw (30:0.4) -- (30:0.6) node(H2)[above right,yshift=-2pt]{\ce{H}};
    
                \draw [rex,thick] (H1) to[out=0,in=130] node[above right,black]{$\theta$} (H2);
            \end{tikzpicture}
            \caption{Viscinal hydrogens.}
            \label{fig:couplingSpecialCaseb}
        \end{subfigure}
        \begin{subfigure}[b]{0.3\linewidth}
            \centering
            \chemfig{@{HB}{H_B}-[:30](=[2](-[:150]@{HA}{H_A})(-[:30,,,,black!20]\textcolor{black!20}{Cl}))-[:-30]@{HC}{H_C}}
            \chemmove{
                \draw [rex,thick,-,shorten >=3pt] (HA) to[bend left=60,looseness=1.3] node[pos=0.75,above right,black]{$J_{AC}$} (HC);
                \draw [rex,thick,-,shorten <=1pt,shorten >=1pt] (HB) to[bend right=30] node[below,black]{$J_{BC}$} (HC);
                \draw [rex,thick,-,shorten <=1pt,shorten >=3pt] (HA) to[bend right=30] node[left,black]{$J_{AB}$} (HB);
            }
            \vspace{2em}
            \caption{Olefins.}
            \label{fig:couplingSpecialCasec}
        \end{subfigure}
        \caption{Special cases in coupling.}
        \label{fig:couplingSpecialCase}
    \end{figure}
    \begin{enumerate}
        \item \emph{cis} vs. \emph{trans} metal complexes.
        \begin{itemize}
            \item $J_\textit{trans}>J_\textit{cis}$.
            \item Example: If we have a square planar coordination compound with four phosphenes, the two pairs of \emph{trans} ligands will (independently) couple more than any of the four pairs of \emph{cis} ligands.
            \item Why: Same argument as \emph{trans} effect; make an MO argument that \emph{trans} will overlap better.
        \end{itemize}
        \item Viscinal coupling: Coupling of atoms separated by two other atoms.
        \begin{itemize}
            \item Example: Two hydrogens on different carbons of ethane.
            \item Coupling magnitude: Depends on the dihedral angle $\theta$ from the Newman projection.
            \item For syn and anti conformers, $J$ will be larger.
            \item For gauche conformers, $J$ will be smaller.
        \end{itemize}
        \item Olefins.
        \begin{itemize}
            \item Example: Vinyl chloride.
            \item $J_{\ce{AC}}=\SIrange{12}{18}{\hertz}$. Sometimes referred to as \emph{trans}.
            \item $J_{\ce{BC}}=\SIrange{0}{3}{\hertz}$. Sometimes referred to as \emph{gem}.
            \item $J_{\ce{AB}}=\SIrange{6}{12}{\hertz}$. Sometimes referred to as \emph{cis}.
        \end{itemize}
    \end{enumerate}
    \item A more realistic view of spins under a magnetic field.
    \begin{itemize}
        \item We may treat the collection of polarized molecules as an ensemble of spins in space.
        \item However, just because a magnetic field has been applied doesn't mean that each nuclear spin is perfectly poised along the $z$-axis.
        \item Rather, there are still residual $x$- and $y$-components that will precess around the $z$-axis.
        \begin{itemize}
            \item We have, in fact, a whole distribution of spins which are arranged in some cone around the $z$-axis. We do indeed have a random and equally dispersed $x,y$-spins.
            \item The precession is called a \textbf{Larmor precession} and occurs at the \textbf{Larmor frequency}.
        \end{itemize}
        % \item Let's capture the fraction of proton spins that are aligned in the $z$-direction.
        % \item These spins are not static; they're precessing about the $z$-axis.
        % \item The precession $\nu_0=-\gamma B_0/2\pi$ is the \textbf{Larmor frequency}.
    \end{itemize}
    \item The NMR experiment.
    \begin{itemize}
        \item What we do is take the ensemble of spins and apply a \ang{90} radiofrequency pulse to project the net magnetization onto the $x$-axis. Then, we monitor the precession rate about the $xy$-plane.
        \item This is called a \textbf{Hahn Echo experiment}.
        \begin{itemize}
            \item We also use this technique in pulsed EPR experiments.
        \end{itemize}
        \item We then Fourier transform the wave to get it back to normal.
        \item Misc. notes.
        \begin{itemize}
            \item The $x$- and $y$-components vary in time.
            \item Both $\alpha$ and $\beta$ (up and down) spins precess at the same frequency.
            \item Chemically equivalent nuclei precess at the same frequency, but are not necessarily in phase.
            \item While the $x$- and $y$-components are scattered, there is still net polarization along $z$ and hence a net magnetization in this direction.
        \end{itemize}
    \end{itemize}
    \item Next time.
    \begin{itemize}
        \item Wikipedia description of the Hahn Echo experiment (really good).
        \item A bit more on NMR, too, to be wrapped up at the beginning of the lecture.
    \end{itemize}
\end{itemize}




\end{document}