\documentclass[../notes.tex]{subfiles}

\pagestyle{main}
\renewcommand{\chaptermark}[1]{\markboth{\chaptername\ \thechapter\ (#1)}{}}
\setcounter{chapter}{6}

\begin{document}




\chapter{???}
\section{EPR Spectroscopy}
\begin{itemize}
    \item \marginnote{2/14:}Today: EPR Spectroscopy.
    \item Thursday: XAS and EXAFS (guest lecturer with great experience in the field).
    \item As an official policy and a cautionary note, don't use ChatGPT to solve this course's homework.
    \item Benefits of EPR.
    \begin{itemize}
        \item Allows you to look at electronic spin flips.
        \item Allows you to look at paramagnetic complexes (the class complementary to those you can observe with NMR).
    \end{itemize}
    \item Most of today: Doublets.
    \item Electron spin.
    \begin{itemize}
        \item An electron can be either spin up or spin down.
        \item These states are degenerate in the absence of a magnetic field.
        \item However, a magnetic field induces a Zeeman\footnote{ZAY-mon.} splitting described as follows.
    \end{itemize}
    \item The intrinsic electron spin has a magnetic moment defined as
    \begin{equation*}
        \mu_z = g\beta M_s
    \end{equation*}
    \begin{itemize}
        \item $g$ is the gyromagnetic ratio for an \emph{electron} ($=2.0023219278$; we usually just treat it as 2; can deviate depending on SOC).
        \item $M_s=\pm 1/2$ is our spin quantum number.
        \item $\beta=e\hbar/2m_e$ is our Bohr magneton.
    \end{itemize}
    \item It follows that
    \begin{equation*}
        E = -\mu\cdot\vec{H}
        = -\mu H\cos(\vec{\mu}\cdot\vec{H})
        = -\mu_zH
        = g\beta HM_s
        = \pm\frac{1}{2}g\beta H
    \end{equation*}
    \begin{itemize}
        \item This Zeeman splitting is pretty easy to visualize (see Figure \ref{fig:zeroFieldSplita}).
        \item Thus, the energy to flip an electron is going to be $g\beta H$.
        \begin{itemize}
            \item This is also known as the \textbf{resonance condition}.
        \end{itemize}
    \end{itemize}
    \item Aside: Be familiar with ENDOR spectroscopy, a combination of nuclear and electronic.
    \item In an EPR experiment, we apply radiation to stimulate a spin flip.
    \begin{itemize}
        \item $E=h\nu=g\beta H$.
        \item This energy is in the microwave region.
        \item Microwaves are much higher in energy than radio waves, thus are more difficult to handle for technical reasons.
        \item Thus, we cannot run complex, pulsed experiments analogous to in NMR; we just run more simple scans.
        \item Indeed, herein we \emph{fix} the microwave frequency and vary the magnetic domain.
        \item Takeaway: EPR is much cruder than NMR in many ways (for technical reasons).
    \end{itemize}
    \item The frequency domain $\nu$ can vary based on the spectrometer from \SIrange{1}{285}{\giga\hertz}.
    \begin{itemize}
        \item X-band (\SI{9.6}{\giga\hertz}) spectrometers are the most common.
        \item More rarely, Q-band (\SI{35}{\giga\hertz}) and W-band (\SI{95}{\giga\hertz}) are used.
        \item Milwaukee has an S-band EPR at \SI{2}{\giga\hertz}.
        \item High frequencies give you worse splitting (opposite of NMR) but better signal-to-noise.
    \end{itemize}
    \item \textbf{Continuous wave EPR}: EPR experiments run under a fixed frequency domain and variable magnetic domain. \emph{Also known as} \textbf{CW EPR}.
    \begin{itemize}
        \item All of the setups described above constitute CW EPR.
    \end{itemize}
    \item We scan across our magnetic field and observe (fairly broad) Gaussian lineshapes. Thus, we usually plot the first derivative.
    \begin{figure}[h!]
        \centering
        \begin{subfigure}[b]{0.45\linewidth}
            \centering
            \begin{tikzpicture}
                \small
                \draw (0,3) -- node[left]{Abs} (0,0) -- node[below]{$H$} (5,0);
    
                \draw [rex,thick] plot[domain=-4:4,smooth,samples=100] ({5*(\x+4)/8},{0.5+1.5*e^(-\x*\x)});
            \end{tikzpicture}
            \caption{Gaussian lineshape.}
            \label{fig:EPR1stDeriva}
        \end{subfigure}
        \begin{subfigure}[b]{0.45\linewidth}
            \centering
            \begin{tikzpicture}
                \small
                \draw (0,3) -- node[left]{Abs} (0,0) -- node[below]{$H$} (5,0);
    
                \draw [rex,thick] plot[domain=-4:4,smooth,samples=100] ({5*(\x+4)/8},{1.5-3*\x*e^(-\x*\x)});
            \end{tikzpicture}
            \caption{First derivative.}
            \label{fig:EPR1stDerivb}
        \end{subfigure}
        \caption{EPR spectra are plotted as the first derivative.}
        \label{fig:EPR1stDeriv}
    \end{figure}
    \item There are two main types of EPR line broadening.
    \begin{enumerate}
        \item \textbf{Secular broadening}. This comes from processes that vary the local magnetic field. Two subclasses.
        \begin{enumerate}
            \item \emph{Dynamic broadening}. A homogeneous broadening obtained by adding a paramagnetic ion to a solution of a free radical. This gives Lorentzian lineshapes.
            \item \emph{Static broadening}. This is effectively the same, but induced by frozen solution and hence no longer homogeneous. This gives Gaussian lineshapes.
        \end{enumerate}
        \item \textbf{Lifetime broadening}. This occurs when the excited state lifetime is too short. It mathematically originates from the Heisenberg uncertainty principle. In particular, we have that
        \begin{align*}
            \Delta E\Delta t &= \frac{h}{2\pi}&
                \Delta h\nu = \Delta E &= g\beta H = g\beta\Delta H\\
            \Delta h\nu\Delta t &= \frac{h}{2\pi}&
                \Delta\nu &= \frac{g\beta}{h}\Delta H\\
            \Delta\nu &= \frac{1}{2\pi\Delta t}&
                \Delta H &= \frac{h}{g\beta}\Delta\nu\\
        \end{align*}
        so, combining the above two expressions, the FWHM $\Delta H$ is given by
        \begin{equation*}
            \Delta H = \frac{h}{2\pi}\frac{1}{g\beta\Delta t}
        \end{equation*}
        Essentially, as $\Delta t$ gets really small, everything else starts to blow up since its in the denominator in the last term.
    \end{enumerate}
    \item We typically do EPR experiments in a frozen solution.
    \item Two flavors of EPR we really care about.
    \item \textbf{Spin-lattice relaxation}: The interaction of spin with the surroundings. \emph{Denoted by} $\bm{T_1}$. \emph{Given by}
    \begin{equation*}
        \Delta H = \frac{\hbar}{g\beta}\frac{1}{2T_1}
    \end{equation*}
    \begin{itemize}
        \item In other words, if some spin is oriented away from $H$, how much time does it take to "reparallelize?"
    \end{itemize}
    \item \textbf{Spin-spin relaxation}. \emph{Denoted by} $\bm{T_2}$. \emph{Given by}
    \begin{equation*}
        \Delta H = \frac{\hbar}{g\beta}\frac{1}{T_2}
    \end{equation*}
    \item Combining the two gives the total homogeneous line width $T_2'$:
    \begin{equation*}
        \frac{1}{T_2}=\frac{1}{T_2}+\frac{1}{2T_1}
    \end{equation*}
    \item Result: We need to go to low temperatures for EPR spectroscopy both for the lifetime reason and for technical reasons.
    \begin{itemize}
        \item We need to go to helium temperatures in many cases.
        \item Low spins may be observable at "high" temperatures like \SI{77}{\kelvin} or even room temperature in some cases.
    \end{itemize}
    \item $g$-values.
    \begin{itemize}
        \item The $g$-value is sort of like a ppm shift value for the EPR signal.
        \item Recall that $g$ is determined by the resonant frequency $H_r$ at which we observe the excitation.
        \begin{align*}
            \Delta E &= \Delta E\\
            g\beta H_r &= h\nu\\
            g &= \frac{h\nu}{\beta H_r}
        \end{align*}
        \item Organic radicals almost always exist at $g=2$ and are isotropic. Deviation might be 1/10,000. This is because organic compounds are composed of light atoms that just don't have much SOC.
        \item For TM's (and lanthanides), $g$ varies widely. Fortunately, it varies in a way that we can understand with orbitals.
        \begin{itemize}
            \item Again, we visualize "ring currents."
            \item One that opposes the magnetic field induces $g<2$.
            \begin{itemize}
                \item Think of a lone electron in a set of 5 degenerate $d$-orbitals.
            \end{itemize}
            \item If a hole "hops" through and generates a ring current as in $d^9$ \ce{Cu^2+}, we have a ring current that reinforces the magnetic field and hence $g>2$.
            \item Essentially, there are three scenarios that determine the value of $g$.
            \begin{itemize}
                \item If the $d$-orbitals are less than 1/2 filled, $g<2$.
                \item If $>$ half filled, $g>2$.
                \item At a perfectly half-filled shell, we should have $g\approx 2$.
            \end{itemize}
            \item This gives us sign. For magnitude\dots
            \begin{itemize}
                \item The magnitude of the deviation is related to the magnitude of the SOC.
                \item To a first approximation, this will relate to symmetry and the degeneracy of orbitals.
            \end{itemize}
            \item Trust the half-filled value, not the hole: When you have $d^2$ in an octahedral field, for instance, the $t_{2g}$ set is less than half-filled, so $g<2$; it's not that you have one hole rotating so $g>2$.
            \item It is the possibility of orbital angular momentum coupling with the spin angular momentum that allows $g$ to change.
        \end{itemize}
    \end{itemize}
    \item Anisotropy: Thus far, we've only considered the effect of isotropic transitions. But both the magnetic field and spin are tensors with $x,y,z$-components.
    \begin{itemize}
        \item An isotropic magnetic field is described by
        \begin{equation*}
            \hat{H}_\text{iso} = g\beta\hat{H}\cdot\hat{S}
        \end{equation*}
        \begin{itemize}
            \item You spin your samples to try to average this out in solid-state NMR, but it's not perfect so you do need to take axes into account.
        \end{itemize}
        \item An anisotropic magnetic field is described by
        \begin{equation*}
            \hat{H}_\text{aniso}=\beta\hat{H}\cdot\hat{g}\cdot\hat{S}=\beta(H_x,H_y,H_z)
            \begin{pmatrix}
                g_x & 0 & 0\\
                0 & g_y & 0\\
                0 & 0 & g_z\\
            \end{pmatrix}
            = \beta H_xg_xS_x+\beta H_yg_yS_y+\beta H_zg_zS_z
        \end{equation*}
        \item If a molecule has very high symmetry (e.g., $T_d$, $O$, $I_h$, etc.), then $g_x=g_y=g_z$. In this case, we get back to our nice isotropic, one peak picture (Figure \ref{fig:EPR1stDeriva}).
        \begin{itemize}
            \item Molecular geometry is \emph{related} to $g$ variations, but they're not the same thing. Don't make the mistake of looking at a highly symmetric molecule and thinking you know what the EPR spectra should be! You have to consider orbitals.
        \end{itemize}
        \item EPR is often done on frozen solutions (sometimes powders) so that we don't have rapid solution-phase tumbling to average out the $g$-values but have everything stuck in some orientation. In particular, $g$-values will be inequivalent in frozen solutions!
    \end{itemize}
    \item Anisotropic spectra.
    \begin{figure}[H]
        \centering
        \begin{subfigure}[b]{0.48\linewidth}
            \centering
            \begin{tikzpicture}
                \small
                \draw (0,3) -- node[left]{Abs} (0,0) -- node[below]{$H$} (5,0);
    
                \draw [rex,thick] plot[domain=-4:4,smooth] ({5*(\x+4)/8},{(0.5+0.2*e^(-5*(\x+1.8)^2))+(0.5+1.5*e^(-(\x)^2))+(0.5+0.2*e^(-5*(\x-1.8)^2))});
            \end{tikzpicture}
            \caption{Gaussian lineshape.}
            \label{fig:EPRanisoa}
        \end{subfigure}
        \begin{subfigure}[b]{0.48\linewidth}
            \centering
            \begin{tikzpicture}
                \small
                \draw (0,3) -- node[left]{Abs} (0,0) -- node[below]{$H$} (5,0);
    
                \draw [rex,thick] plot[domain=-4:4,smooth,samples=100] ({5*(\x+4)/8},{(0.5-2*(\x+1.8)*e^(-5*(\x+1.8)^2))+(0.5-3*\x*e^(-(\x)^2))+(0.5-2*(\x-1.8)*e^(-5*(\x-1.8)^2))});

                \footnotesize
                \node at (1.201,1.5) {$g_x$};
                \node at (2.064,1.5) {$g_y$};
                \node at (3.799,1.5) {$g_z$};
            \end{tikzpicture}
            \caption{First derivative.}
            \label{fig:EPRanisob}
        \end{subfigure}
        \caption{Rhombic anisotroic EPR spectra.}
        \label{fig:EPRaniso}
    \end{figure}
    \begin{itemize}
        \item Note that we don't know anything about Cartesian coordinates from the spectrum, so Anderson prefers $g_1,g_2,g_3$.
        \item This is a rhombic spectrum, meaning that $g_x\neq g_y\neq g_z$.
        \item Axial spectra are often referred to as $g_\parallel$ and $g_\perp$. This corresponds to $g_x\approx g_y\neq g_z$.
        \begin{itemize}
            \item We typically have a small $g_\parallel$ peak downfield of a big $g_\perp$ peak (picture in notes).
            \item You see spectra like this for when your molecules have a clear symmetry axis; decreasing the symmetry further will tend to introduce some more rhombicity.
        \end{itemize}
        \item Anisotropy only occurs via coupling into orbital angular momentum. Thus, it is observed for TMs but typically not for organic radicals.
    \end{itemize}
    \item We now discuss fine structure for a while.
    \item Fine structure basics.
    \begin{itemize}
        \item The electron spin can also interact with nuclear spin in \textbf{hyperfine interactions} and \textbf{superhyperfine interactions}.
        \begin{itemize}
            \item These lead to additional signals and structure.
            \item Be aware of the differences between hyperfine and superhyperfine, but only hyperfine interactions will be analyzed for the remainder of the time.
        \end{itemize}
        \item The quantum-mechanical basis for fine structure involves the following Hamiltonian.
        \begin{equation*}
            \hat{H} = g\beta\vec{H}\cdot\hat{S}+a\hat{I}\cdot\hat{S}
            = g\beta H\cdot\hat{S}_z+a\hat{I}_z\cdot\hat{S}_z
        \end{equation*}
        \begin{itemize}
            \item $\hat{I}$ is the nuclear spin operator.
            \item $\hat{S}$ is the electron spin operator.
            \item $z$ is the component along $H$.
            \item $a$ is the hyperfine coupling constant.
        \end{itemize}
        \item Energies: Found by calculating eigenvalues.
        \begin{equation*}
            E = \pm\frac{1}{2}g\beta H+\frac{a}{4}
            = \frac{1}{2}g\beta H\pm\frac{a}{4}
        \end{equation*}
    \end{itemize}
    \item \textbf{Hyperfine interaction}: An interaction with the direct nucleus which bears the radical.
    \item \textbf{Superhyperfine interaction}: An interaction involving additional coupling to other nuclei via delocalization.
    \item Hyperfine Zeeman splitting diagram.
    \begin{figure}[h!]
        \centering
        \begin{tikzpicture}[
            every node/.style={black}
        ]
            \small
            \draw (0,3) node[above]{$E$} -- (0,0) -- (5,0) node[right]{$H$};
    
            \footnotesize
            \draw (1.5,0.1) -- ++(0,-0.2) node[below]{$\e[-]$};
            \draw (2.75,0.1) -- ++(0,-0.2) node[below]{nuclear};
    
            \draw [blx,ultra thick]
                (0,1.5) -- ++(0.5,0)
                (1.25,2.3) -- ++(0.5,0) node[above left]{$M_s=+\frac{1}{2}$}
                (1.25,0.7) -- ++(0.5,0) node[below left]{$M_s=-\frac{1}{2}$}
                (2.5,2.6) -- ++(0.5,0)
                (2.5,2) -- ++(0.5,0)
                (2.5,1) -- ++(0.5,0)
                (2.5,0.4) -- ++(0.5,0)
            ;
            \draw [blx,semithick]
                (0.5,1.5) -- (1.25,2.3)
                (0.5,1.5) -- (1.25,0.7)
                (1.75,2.3) -- (2.5,2.6)
                (1.75,2.3) -- (2.5,2)
                (1.75,0.7) -- (2.5,1)
                (1.75,0.7) -- (2.5,0.4)
            ;
            \draw [rex,thick,stealth-stealth,shorten <=1pt,shorten >=1pt] (2.65,0.4) -- (2.65,2.6);
            \draw [orx,thick,stealth-stealth,shorten <=1pt,shorten >=1pt] (2.85,1) -- (2.85,2);
    
            \node at (3.5,3.1) {$M_s$};
            \node at (3.5,2.6) {$+\frac{1}{2}$};
            \node at (3.5,2) {$+\frac{1}{2}$};
            \node at (3.5,1) {$-\frac{1}{2}$};
            \node at (3.5,0.4) {$-\frac{1}{2}$};
    
            \node at (4.5,3.1) {$M_I$};
            \node at (4.5,2.6) {$+\frac{1}{2}$};
            \node at (4.5,2) {$-\frac{1}{2}$};
            \node at (4.5,1) {$-\frac{1}{2}$};
            \node at (4.5,0.4) {$+\frac{1}{2}$};
        \end{tikzpicture}
        \caption{Hyperfine splitting.}
        \label{fig:hyperfineSplitting}
    \end{figure}
    \begin{itemize}
        \item We first have our electron splitting.
        \item Then if we turn on nuclear splitting, that happens next, yielding four distinct states.
    \end{itemize}
    \item Hyperfine election rules.
    \begin{itemize}
        \item Thus far, we have implicitly used $\Delta M_s=\pm 1$.
        \item We must have $\Delta M_I=0$ as well! We can have an electron spin flip, but we can't do a nuclear spin flip at the same time, which may make some intuitive sense as well since the probability that these two unlikely events would happen simultaneously is very low.
        \item Thus, we have two allowed transitions in Figure \ref{fig:hyperfineSplitting}.
    \end{itemize}
    \item Hyperfine EPR spectrum.
    \begin{itemize}
        \item The peak splitting is $A$, which is related to the actual hyperfine coupling constant $a$ as we'll talk about shortly.
        \item A lot of the time, you don't see $y$-axis labels in EPR spectra because "the first derivative of absorbance" is not a particularly helpful unit.
    \end{itemize}
    \item A few final notes on hyperfine coupling.
    \begin{enumerate}
        \item The hyperfine coupling constant $a$ is field independent.
        \begin{itemize}
            \item Zeeman splitting is the only thing that is field dependent.
        \end{itemize}
        \item $a$ is a scalar HF constant and
        \begin{equation*}
            a = g\beta A
        \end{equation*}
        \begin{itemize}
            \item $A$ is measured in Gauss; it is the \emph{experimental} HF constant.
        \end{itemize}
        \item $a$ can also be anisotropic.
        \begin{itemize}
            \item Each of the values $g_x,g_y,g_z$ can have an $a$.
            \item $a$ can be 0 for some $g_i$ and not others.
        \end{itemize}
        \item $M_I$ can have many values.
        \begin{itemize}
            \item Just like with NMR, a nuclear spin with $M_I>1/2$ can give multiline patterns.
            \item Examples.
            \begin{itemize}
                \item \ce{{}^14N} has $M_I>1/2$.
                \item TMs can go up to 7/2 or 9/2, leading to very complicated multiline coupling patterns very quickly.
            \end{itemize}
        \end{itemize}
        \item $M_I$ sums the contribution from other neighboring nuclei.
        \begin{itemize}
            \item Multiple nuclei can also give multiline patterns, notably with $2nI+1$ lines.
        \end{itemize}
        \item If a spin active nucleus is not 100\% abundant, we get a superposition.
        \begin{itemize}
            \item Cool example: \ce{Mo(CN)8^3-}.
            \begin{itemize}
                \item This is a $d^1$ $S=1/2$ compound.
                \item $I=0$ is 75\% abundant, but \ce{{}^97Mo} is 25\% abundant with $I=5/2$.
                \item Thus, an amount of intensity equivalent to 1/3 of the central signal gets divided among 6 smaller neighboring peaks that are all evenly spaced apart (see notes for image).
            \end{itemize}
        \end{itemize}
    \end{enumerate}
    \item We now get move on to higher spin species.
    \begin{itemize}
        \item This is a further level of complication.
        \item However, these species may have more accessible EPR transitions.
        \item We usually have larger splitting for higher $M_s$ values.
    \end{itemize}
    \item \textbf{Kramer's theorem}: Ions with an odd number of electrons will always have as their lowest energy level at least a doublet that will split in an applied magnetic field and give rise to an EPR signal.
    \item \textbf{Kramer's doublet}: The doublet predicted by Kramer's theorem.
    \item Example: Transitions predicted to be visible for a quartet ($S=3/2$).
    \begin{figure}[H]
        \centering
        \begin{tikzpicture}[
            every node/.style={black}
        ]
            \small
            \draw (0,4) node[above]{$E$} -- (0,0) -- (4,0) node[right]{$H$ (\si{\gauss})};
    
            \footnotesize
            \draw (1,0.1) -- ++(0,-0.2) node[below]{3000};
            \draw (2,0.1) -- ++(0,-0.2) node[below]{6000};
            \draw (3,0.1) -- ++(0,-0.2) node[below]{9000};
    
            \draw [blx,ultra thick]
                (0,3) node[left]{$M_s=+\frac{3}{2}$} -- ++(0.5,0)
                (0,1) node[left]{$M_s=-\frac{3}{2}$} -- ++(0.5,0)
            ;
            \draw [blx,semithick,name path={4}] (0.5,3) -- ++(3.3,1);
            \draw [blx,semithick,name path={3}] (0.5,3) -- ++(3.3,-1);
            \draw [blx,semithick,name path={2}] (0.5,1) -- ++(3.3,0.7);
            \draw [blx,semithick,name path={1}] (0.5,1) -- ++(3.3,-0.7);
    
            \path [name path={2000}] ({2/3},0) -- ++(0,4);
            \draw [rex,thick,name intersections={of=2000 and 3,by=a},name intersections={of=2000 and 4,by=b}] (a) -- (b);
    
            \path [name path={3000}] (1,0) -- ++(0,4);
            \draw [rex,thick,name intersections={of=3000 and 1,by=a},name intersections={of=3000 and 2,by=b}] (a) -- (b);
    
            \path [name path={9000}] (3,0) -- ++(0,4);
            \draw [rex,thick,name intersections={of=9000 and 2,by=a},name intersections={of=9000 and 3,by=b}] (a) -- (b);
        \end{tikzpicture}
        \caption{Quartet EPR signals.}
        \label{fig:EPRquartet}
    \end{figure}
    \begin{itemize}
        \item Something lower around \SI{2000}{\gauss}, higher around \SI{3000}{\gauss} and, if our spectrometer is powerful enough, something around \SI{9000}{\gauss}.
        \item Recall $\Delta M_s=\pm 1$.
    \end{itemize}
    \item Example: Transitions predicted to be visible for a triplet ($S=1$).
    \begin{itemize}
        \item Triplets are very difficult to see because the lower state takes a long time to drop low enough to see the ground state.
    \end{itemize}
    \item Treating spin-orbit coupling.
    \begin{itemize}
        \item We have that $\vec{\mu}=g\beta\vec{J}$.
        \item It follows that
        \begin{equation*}
            g = \frac{J(J+1)\cdot S(S+1)-L(L+1)}{2J(J+1)}+1
        \end{equation*}
        \item What people tend to do is look at the spectrum.
    \end{itemize}
    \item That's it for higher spin species.
    \item A few practical points on EPR spectroscopy.
    \begin{enumerate}
        \item Use a microwave cavity, a fixed frequency, and vary the field.
        \item Avoid solvents that strongly absorb microwaves (e.g., water, other polar solvents). This can sometimes be ok if the sample is frozen.
        \item Usually quote $g$-values.
        \begin{itemize}
            \item Do we report values as $H$ or $g$??
        \end{itemize}
        \item You can double integrate vs. a standard to quantity $\pm 20\%$.
    \end{enumerate}
    \item We now move on to advanced EPR techniques.
    \begin{itemize}
        \item These are analogous to some of the fancy things you can do with NMR.
        \item Essentially, they are alternate ways of getting data out of an EPR machine besides straight-up absorption spectra.
    \end{itemize}
    \item In particular, we will focus on \textbf{ENDOR} and \textbf{ESEEM} spectroscopy.
    \item \textbf{Electron spin echo envelope modulation} (spectroscopy). \emph{Also known as} \textbf{ESEEM}.
    \item \textbf{Electron nuclear double resonance} (spectroscopy). \emph{Also known as} \textbf{ENDOR}.
    \item Background.
    \begin{itemize}
        \item Most EPR spectroscopy compared to NMR is pretty rudimentary.
        \begin{itemize}
            \item In NMR, every carbon spectrum you take is proton-decoupled (we'll talk about this later).
            \item Radiofrequency is really nice to control.
        \end{itemize}
        \item However, these are both 2D experiments between EPR and NMR.
        \begin{itemize}
            \item They allow us to see spin-active nuclei that are close to the paramagnetic center.
            \item 2D EPR can give you a ton of information about paramagnetic compounds; the most useful are those that couple electron transitions with nuclear transitions.
        \end{itemize}
        \item These are super powerful techniques.
    \end{itemize}
    \item ENDOR spectroscopy.
    \begin{figure}[h!]
        \centering
        \begin{tikzpicture}[
            every node/.style={black}
        ]
            \small
            \draw (0,3) node[above]{$E$} -- (0,0) -- (5,0) node[right]{$H$};
    
            \footnotesize
            \draw (1.5,0.1) -- ++(0,-0.2) node[below]{EPR};
            \draw (2.75,0.1) -- ++(0,-0.2) node[below]{NMR};
            \draw (4,0.1) -- ++(0,-0.2) node[below]{HF};
    
            \draw [blx,ultra thick]
                (0,1.5) -- ++(0.5,0)
                (1.25,2.3) -- ++(0.5,0) node[above left]{$+\frac{1}{2}g\beta H$}
                (1.25,0.7) -- ++(0.5,0) node[below left]{$-\frac{1}{2}g\beta H$}
                (2.5,2.6) -- ++(0.5,0)
                (2.5,2) -- ++(0.5,0)
                (2.5,1) -- ++(0.5,0)
                (2.5,0.4) -- ++(0.5,0)
                (3.75,2.5) -- ++(0.9,0)
                (3.75,2.1) -- ++(0.9,0)
                (3.75,1.1) -- ++(0.9,0)
                (3.75,0.3) -- ++(0.9,0)
            ;
            \draw [blx,semithick]
                (0.5,1.5) -- (1.25,2.3)
                (0.5,1.5) -- (1.25,0.7)
                (1.75,2.3) -- (2.5,2.6)
                (1.75,2.3) -- (2.5,2)
                (1.75,0.7) -- (2.5,1)
                (1.75,0.7) -- (2.5,0.4)
                (3,2.6) -- (3.75,2.5)
                (3,2) -- (3.75,2.1)
                (3,1) -- (3.75,1.1)
                (3,0.4) -- (3.75,0.3)
            ;
            \draw [rex,thick,-stealth,shorten <=1pt,shorten >=1pt] (3.9,0.3) -- (3.9,2.1);
            \draw [rex,thick,-stealth,shorten <=1pt,shorten >=1pt] (4.1,1.1) -- (4.1,2.5);
            \draw [orx,thick,-stealth,shorten <=1pt,shorten >=1pt] (4.1,0.3) -- (4.1,1.1);
            \draw [orx,thick,-stealth,shorten <=1pt,shorten >=1pt] (3.9,2.1) -- (3.9,2.5);
            \draw [ylx,thick,-stealth,shorten <=1pt,shorten >=1pt] (4.3,1.1) -- (4.3,2.1);
            \draw [ylx,thick,-stealth,shorten <=1pt,shorten >=1pt] (4.5,0.3) -- (4.5,2.5);
    
            \draw [rex,ultra thick] (5,1.1) -- ++(0.5,0) node[right]{EPR};
            \draw [orx,ultra thick] (5,0.8) -- ++(0.5,0) node[right]{NMR};
            \draw [ylx,ultra thick] (5,0.5) -- ++(0.5,0) node[right]{ENDOR};
        \end{tikzpicture}
        \caption{ENDOR splitting.}
        \label{fig:ENDORsplitting}
    \end{figure}
    \begin{itemize}
        \item The quantum-mechanical basis for ENDOR involves the following Hamiltonian.
        \begin{equation*}
            \hat{H} = \underbrace{\vphantom{g_N}g\beta\hat{H}\cdot\hat{S}}_\text{EPR}-\underbrace{g_N\beta_N\hat{H}\cdot\hat{I}}_\text{NMR}+\underbrace{\vphantom{g_N}ha\hat{I}\cdot\hat{S}}_\text{HF}
        \end{equation*}
        \item Three levels of splitting: EPR, NMR, and HF.
        \begin{itemize}
            \item The last one doesn't actually induce splitting; it's just coupling between the first two that affects energy levels.
        \end{itemize}
        \item EPR has $\Delta M_s=\pm 1$, $\Delta M_I=0$.
        \item Here, we turn on semi-forbidden transitions, i.e., those with $\Delta M_s=\pm 1$ and $\Delta M_I=\pm 1$.
        \item $1\to 3$ and $2\to 4$ are EPR allowed.
        \item $1\to 2$ and $3\to 4$ are NMR allowed.
        \item ENDOR transitions (which we're turning on in this experiment) are $2\to 3$ and $1\to 4$.
        \item ENDOR spectra tend to look like look like a big messy hunk corresponding to the nucleus, and then distinct ones further out which correspond to the NMR transitions.
    \end{itemize}
    \item Possibly a bit more ENDOR on Friday.
    \item We now move on to ESEEM.
    \begin{itemize}
        \item Do we need to know this??
    \end{itemize}
\end{itemize}




\end{document}