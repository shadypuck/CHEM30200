\documentclass[../notes.tex]{subfiles}

\pagestyle{main}
\renewcommand{\chaptermark}[1]{\markboth{\chaptername\ \thechapter\ (#1)}{}}
\setcounter{chapter}{6}

\begin{document}




\chapter{EPR and XAS}
\section{EPR Spectroscopy}
\begin{itemize}
    \item \marginnote{2/14:}Today: EPR Spectroscopy.
    \item Thursday: XAS and EXAFS (guest lecturer with great experience in the field).
    \item As an official policy and a cautionary note, don't use ChatGPT to solve this course's homework.
    \item Benefits of EPR.
    \begin{itemize}
        \item Allows you to look at electronic spin flips.
        \item Allows you to look at paramagnetic complexes (the class complementary to those you can observe with NMR).
    \end{itemize}
    \item Most of today: Doublets.
    \item Electron spin.
    \begin{itemize}
        \item An electron can be either spin up or spin down.
        \item These states are degenerate in the absence of a magnetic field.
        \item However, a magnetic field induces a Zeeman\footnote{ZAY-mon.} splitting described as follows.
    \end{itemize}
    \item The intrinsic electron spin has a magnetic moment defined as
    \begin{equation*}
        \mu_z = g\beta M_s
    \end{equation*}
    \begin{itemize}
        \item $g$ is the gyromagnetic ratio for an \emph{electron} ($=2.0023219278$; we usually just treat it as 2; can deviate depending on SOC).
        \item $M_s=\pm 1/2$ is our spin quantum number.
        \item $\beta=e\hbar/2m_e$ is our Bohr magneton.
    \end{itemize}
    \item It follows that
    \begin{equation*}
        E = -\mu\cdot\vec{H}
        = -\mu H\cos(\vec{\mu}\cdot\vec{H})
        = -\mu_zH
        = g\beta HM_s
        = \pm\frac{1}{2}g\beta H
    \end{equation*}
    \begin{itemize}
        \item This Zeeman splitting is pretty easy to visualize (see Figure \ref{fig:zeroFieldSplita}).
        \item Thus, the energy to flip an electron is going to be $g\beta H$.
        \begin{itemize}
            \item This is also known as the \textbf{resonance condition}.
        \end{itemize}
    \end{itemize}
    \item Aside: Be familiar with ENDOR spectroscopy, a combination of nuclear and electronic.
    \item In an EPR experiment, we apply radiation to stimulate a spin flip.
    \begin{itemize}
        \item $E=h\nu=g\beta H$.
        \item This energy is in the microwave region.
        \item Microwaves are much higher in energy than radio waves, thus are more difficult to handle for technical reasons.
        \item Thus, we cannot run complex, pulsed experiments analogous to in NMR; we just run more simple scans.
        \item Indeed, herein we \emph{fix} the microwave frequency and vary the magnetic domain.
        \item Takeaway: EPR is much cruder than NMR in many ways (for technical reasons).
    \end{itemize}
    \item The frequency domain $\nu$ can vary based on the spectrometer from \SIrange{1}{285}{\giga\hertz}.
    \begin{itemize}
        \item X-band (\SI{9.6}{\giga\hertz}) spectrometers are the most common.
        \item More rarely, Q-band (\SI{35}{\giga\hertz}) and W-band (\SI{95}{\giga\hertz}) are used.
        \item Milwaukee has an S-band EPR at \SI{2}{\giga\hertz}.
        \item High frequencies give you worse splitting (opposite of NMR) but better signal-to-noise.
    \end{itemize}
    \item \textbf{Continuous wave EPR}: EPR experiments run under a fixed frequency domain and variable magnetic domain. \emph{Also known as} \textbf{CW EPR}.
    \begin{itemize}
        \item All of the setups described above constitute CW EPR.
    \end{itemize}
    \item We scan across our magnetic field and observe (fairly broad) Gaussian lineshapes. Thus, we usually plot the first derivative.
    \begin{figure}[h!]
        \centering
        \begin{subfigure}[b]{0.45\linewidth}
            \centering
            \begin{tikzpicture}
                \small
                \draw (0,3) -- node[left]{Abs} (0,0) -- node[below]{$H$} (5,0);
    
                \draw [rex,thick] plot[domain=-4:4,smooth,samples=100] ({5*(\x+4)/8},{0.5+1.5*e^(-\x*\x)});
            \end{tikzpicture}
            \caption{Gaussian lineshape.}
            \label{fig:EPR1stDeriva}
        \end{subfigure}
        \begin{subfigure}[b]{0.45\linewidth}
            \centering
            \begin{tikzpicture}
                \small
                \draw (0,3) -- node[left]{Abs} (0,0) -- node[below]{$H$} (5,0);
    
                \draw [rex,thick] plot[domain=-4:4,smooth,samples=100] ({5*(\x+4)/8},{1.5-3*\x*e^(-\x*\x)});
            \end{tikzpicture}
            \caption{First derivative.}
            \label{fig:EPR1stDerivb}
        \end{subfigure}
        \caption{EPR spectra are plotted as the first derivative.}
        \label{fig:EPR1stDeriv}
    \end{figure}
    \item There are two main types of EPR line broadening.
    \begin{enumerate}
        \item \textbf{Secular broadening}. This comes from processes that vary the local magnetic field. Two subclasses.
        \begin{enumerate}
            \item \emph{Dynamic broadening}. A homogeneous broadening obtained by adding a paramagnetic ion to a solution of a free radical. This gives Lorentzian lineshapes.
            \item \emph{Static broadening}. This is effectively the same, but induced by frozen solution and hence no longer homogeneous. This gives Gaussian lineshapes.
        \end{enumerate}
        \item \textbf{Lifetime broadening}. This occurs when the excited state lifetime is too short. It mathematically originates from the Heisenberg uncertainty principle. In particular, we have that
        \begin{align*}
            \Delta E\Delta t &= \frac{h}{2\pi}&
                \Delta h\nu = \Delta E &= g\beta H = g\beta\Delta H\\
            \Delta h\nu\Delta t &= \frac{h}{2\pi}&
                \Delta\nu &= \frac{g\beta}{h}\Delta H\\
            \Delta\nu &= \frac{1}{2\pi\Delta t}&
                \Delta H &= \frac{h}{g\beta}\Delta\nu\\
        \end{align*}
        so, combining the above two expressions, the FWHM $\Delta H$ is given by
        \begin{equation*}
            \Delta H = \frac{h}{2\pi}\frac{1}{g\beta\Delta t}
        \end{equation*}
        Essentially, as $\Delta t$ gets really small, everything else starts to blow up since its in the denominator in the last term.
    \end{enumerate}
    \item We typically do EPR experiments in a frozen solution.
    \item Two flavors of EPR we really care about.
    \item \textbf{Spin-lattice relaxation}: The interaction of spin with the surroundings. \emph{Denoted by} $\bm{T_1}$. \emph{Given by}
    \begin{equation*}
        \Delta H = \frac{\hbar}{g\beta}\frac{1}{2T_1}
    \end{equation*}
    \begin{itemize}
        \item In other words, if some spin is oriented away from $H$, how much time does it take to "reparallelize?"
    \end{itemize}
    \item \textbf{Spin-spin relaxation}. \emph{Denoted by} $\bm{T_2}$. \emph{Given by}
    \begin{equation*}
        \Delta H = \frac{\hbar}{g\beta}\frac{1}{T_2}
    \end{equation*}
    \item Combining the two gives the total homogeneous line width $T_2'$:
    \begin{equation*}
        \frac{1}{T_2}=\frac{1}{T_2}+\frac{1}{2T_1}
    \end{equation*}
    \item Result: We need to go to low temperatures for EPR spectroscopy both for the lifetime reason and for technical reasons.
    \begin{itemize}
        \item We need to go to helium temperatures in many cases.
        \item Low spins may be observable at "high" temperatures like \SI{77}{\kelvin} or even room temperature in some cases.
    \end{itemize}
    \item $g$-values.
    \begin{itemize}
        \item The $g$-value is sort of like a ppm shift value for the EPR signal.
        \item Recall that $g$ is determined by the resonant frequency $H_r$ at which we observe the excitation.
        \begin{align*}
            \Delta E &= \Delta E\\
            g\beta H_r &= h\nu\\
            g &= \frac{h\nu}{\beta H_r}
        \end{align*}
        \item Organic radicals almost always exist at $g=2$ and are isotropic. Deviation might be 1/10,000. This is because organic compounds are composed of light atoms that just don't have much SOC.
        \item For TM's (and lanthanides), $g$ varies widely. Fortunately, it varies in a way that we can understand with orbitals.
        \begin{itemize}
            \item Again, we visualize "ring currents."
            \item One that opposes the magnetic field induces $g<2$.
            \begin{itemize}
                \item Think of a lone electron in a set of 5 degenerate $d$-orbitals.
            \end{itemize}
            \item If a hole "hops" through and generates a ring current as in $d^9$ \ce{Cu^2+}, we have a ring current that reinforces the magnetic field and hence $g>2$.
            \item Essentially, there are three scenarios that determine the value of $g$.
            \begin{itemize}
                \item If the $d$-orbitals are less than 1/2 filled, $g<2$.
                \item If $>$ half filled, $g>2$.
                \item At a perfectly half-filled shell, we should have $g\approx 2$.
            \end{itemize}
            \item This gives us sign. For magnitude\dots
            \begin{itemize}
                \item The magnitude of the deviation is related to the magnitude of the SOC.
                \item To a first approximation, this will relate to symmetry and the degeneracy of orbitals.
            \end{itemize}
            \item Trust the half-filled value, not the hole: When you have $d^2$ in an octahedral field, for instance, the $t_{2g}$ set is less than half-filled, so $g<2$; it's not that you have one hole rotating so $g>2$.
            \item It is the possibility of orbital angular momentum coupling with the spin angular momentum that allows $g$ to change.
        \end{itemize}
    \end{itemize}
    \item Anisotropy: Thus far, we've only considered the effect of isotropic transitions. But both the magnetic field and spin are tensors with $x,y,z$-components.
    \begin{itemize}
        \item An isotropic magnetic field is described by
        \begin{equation*}
            \hat{H}_\text{iso} = g\beta\hat{H}\cdot\hat{S}
        \end{equation*}
        \begin{itemize}
            \item You spin your samples to try to average this out in solid-state NMR, but it's not perfect so you do need to take axes into account.
        \end{itemize}
        \item An anisotropic magnetic field is described by
        \begin{equation*}
            \hat{H}_\text{aniso}=\beta\hat{H}\cdot\hat{g}\cdot\hat{S}=\beta(H_x,H_y,H_z)
            \begin{pmatrix}
                g_x & 0 & 0\\
                0 & g_y & 0\\
                0 & 0 & g_z\\
            \end{pmatrix}
            = \beta H_xg_xS_x+\beta H_yg_yS_y+\beta H_zg_zS_z
        \end{equation*}
        \item If a molecule has very high symmetry (e.g., $T_d$, $O$, $I_h$, etc.), then $g_x=g_y=g_z$. In this case, we get back to our nice isotropic, one peak picture (Figure \ref{fig:EPR1stDeriva}).
        \begin{itemize}
            \item Molecular geometry is \emph{related} to $g$ variations, but they're not the same thing. Don't make the mistake of looking at a highly symmetric molecule and thinking you know what the EPR spectra should be! You have to consider orbitals.
        \end{itemize}
        \item EPR is often done on frozen solutions (sometimes powders) so that we don't have rapid solution-phase tumbling to average out the $g$-values but have everything stuck in some orientation. In particular, $g$-values will be inequivalent in frozen solutions!
    \end{itemize}
    \item Anisotropic spectra.
    \begin{figure}[H]
        \centering
        \begin{subfigure}[b]{0.48\linewidth}
            \centering
            \begin{tikzpicture}
                \small
                \draw (0,3) -- node[left]{Abs} (0,0) -- node[below]{$H$} (5,0);
    
                \draw [rex,thick] plot[domain=-4:4,smooth] ({5*(\x+4)/8},{(0.5+0.2*e^(-5*(\x+1.8)^2))+(0.5+1.5*e^(-(\x)^2))+(0.5+0.2*e^(-5*(\x-1.8)^2))});
            \end{tikzpicture}
            \caption{Gaussian lineshape.}
            \label{fig:EPRanisoa}
        \end{subfigure}
        \begin{subfigure}[b]{0.48\linewidth}
            \centering
            \begin{tikzpicture}
                \small
                \draw (0,3) -- node[left]{Abs} (0,0) -- node[below]{$H$} (5,0);
    
                \draw [rex,thick] plot[domain=-4:4,smooth,samples=100] ({5*(\x+4)/8},{(0.5-2*(\x+1.8)*e^(-5*(\x+1.8)^2))+(0.5-3*\x*e^(-(\x)^2))+(0.5-2*(\x-1.8)*e^(-5*(\x-1.8)^2))});

                \footnotesize
                \node at (1.201,1.5) {$g_x$};
                \node at (2.064,1.5) {$g_y$};
                \node at (3.799,1.5) {$g_z$};
            \end{tikzpicture}
            \caption{First derivative.}
            \label{fig:EPRanisob}
        \end{subfigure}
        \caption{Rhombic anisotroic EPR spectra.}
        \label{fig:EPRaniso}
    \end{figure}
    \begin{itemize}
        \item Note that we don't know anything about Cartesian coordinates from the spectrum, so Anderson prefers $g_1,g_2,g_3$.
        \item This is a rhombic spectrum, meaning that $g_x\neq g_y\neq g_z$.
        \item Axial spectra are often referred to as $g_\parallel$ and $g_\perp$. This corresponds to $g_x\approx g_y\neq g_z$.
        \begin{itemize}
            \item We typically have a small $g_\parallel$ peak downfield of a big $g_\perp$ peak (picture in notes).
            \item You see spectra like this for when your molecules have a clear symmetry axis; decreasing the symmetry further will tend to introduce some more rhombicity.
        \end{itemize}
        \item Anisotropy only occurs via coupling into orbital angular momentum. Thus, it is observed for TMs but typically not for organic radicals.
    \end{itemize}
    \item We now discuss fine structure for a while.
    \item Fine structure basics.
    \begin{itemize}
        \item The electron spin can also interact with nuclear spin in \textbf{hyperfine interactions} and \textbf{superhyperfine interactions}.
        \begin{itemize}
            \item These lead to additional signals and structure.
            \item Be aware of the differences between hyperfine and superhyperfine, but only hyperfine interactions will be analyzed for the remainder of the time.
        \end{itemize}
        \item The quantum-mechanical basis for fine structure involves the following Hamiltonian.
        \begin{equation*}
            \hat{H} = g\beta\vec{H}\cdot\hat{S}+a\hat{I}\cdot\hat{S}
            = g\beta H\cdot\hat{S}_z+a\hat{I}_z\cdot\hat{S}_z
        \end{equation*}
        \begin{itemize}
            \item $\hat{I}$ is the nuclear spin operator.
            \item $\hat{S}$ is the electron spin operator.
            \item $z$ is the component along $H$.
            \item $a$ is the hyperfine coupling constant.
        \end{itemize}
        \item Energies: Found by calculating eigenvalues.
        \begin{equation*}
            E = \pm\frac{1}{2}g\beta H+\frac{a}{4}
            = \frac{1}{2}g\beta H\pm\frac{a}{4}
        \end{equation*}
    \end{itemize}
    \item \textbf{Hyperfine interaction}: An interaction with the direct nucleus which bears the radical.
    \item \textbf{Superhyperfine interaction}: An interaction involving additional coupling to other nuclei via delocalization.
    \item Hyperfine Zeeman splitting diagram.
    \begin{figure}[h!]
        \centering
        \begin{tikzpicture}[
            every node/.style={black}
        ]
            \small
            \draw (0,3) node[above]{$E$} -- (0,0) -- (5,0) node[right]{$H$};
    
            \footnotesize
            \draw (1.5,0.1) -- ++(0,-0.2) node[below]{$\e[-]$};
            \draw (2.75,0.1) -- ++(0,-0.2) node[below]{nuclear};
    
            \draw [blx,ultra thick]
                (0,1.5) -- ++(0.5,0)
                (1.25,2.3) -- ++(0.5,0) node[above left]{$M_s=+\frac{1}{2}$}
                (1.25,0.7) -- ++(0.5,0) node[below left]{$M_s=-\frac{1}{2}$}
                (2.5,2.6) -- ++(0.5,0)
                (2.5,2) -- ++(0.5,0)
                (2.5,1) -- ++(0.5,0)
                (2.5,0.4) -- ++(0.5,0)
            ;
            \draw [blx,semithick]
                (0.5,1.5) -- (1.25,2.3)
                (0.5,1.5) -- (1.25,0.7)
                (1.75,2.3) -- (2.5,2.6)
                (1.75,2.3) -- (2.5,2)
                (1.75,0.7) -- (2.5,1)
                (1.75,0.7) -- (2.5,0.4)
            ;
            \draw [rex,thick,stealth-stealth,shorten <=1pt,shorten >=1pt] (2.65,0.4) -- (2.65,2.6);
            \draw [orx,thick,stealth-stealth,shorten <=1pt,shorten >=1pt] (2.85,1) -- (2.85,2);
    
            \node at (3.5,3.1) {$M_s$};
            \node at (3.5,2.6) {$+\frac{1}{2}$};
            \node at (3.5,2) {$+\frac{1}{2}$};
            \node at (3.5,1) {$-\frac{1}{2}$};
            \node at (3.5,0.4) {$-\frac{1}{2}$};
    
            \node at (4.5,3.1) {$M_I$};
            \node at (4.5,2.6) {$+\frac{1}{2}$};
            \node at (4.5,2) {$-\frac{1}{2}$};
            \node at (4.5,1) {$-\frac{1}{2}$};
            \node at (4.5,0.4) {$+\frac{1}{2}$};
        \end{tikzpicture}
        \caption{Hyperfine splitting.}
        \label{fig:hyperfineSplitting}
    \end{figure}
    \begin{itemize}
        \item We first have our electron splitting.
        \item Then if we turn on nuclear splitting, that happens next, yielding four distinct states.
    \end{itemize}
    \item Hyperfine election rules.
    \begin{itemize}
        \item Thus far, we have implicitly used $\Delta M_s=\pm 1$.
        \item We must have $\Delta M_I=0$ as well! We can have an electron spin flip, but we can't do a nuclear spin flip at the same time, which may make some intuitive sense as well since the probability that these two unlikely events would happen simultaneously is very low.
        \item Thus, we have two allowed transitions in Figure \ref{fig:hyperfineSplitting}.
    \end{itemize}
    \item Hyperfine EPR spectrum.
    \begin{itemize}
        \item The peak splitting is $A$, which is related to the actual hyperfine coupling constant $a$ as we'll talk about shortly.
        \item A lot of the time, you don't see $y$-axis labels in EPR spectra because "the first derivative of absorbance" is not a particularly helpful unit.
    \end{itemize}
    \item A few final notes on hyperfine coupling.
    \begin{enumerate}
        \item The hyperfine coupling constant $a$ is field independent.
        \begin{itemize}
            \item Zeeman splitting is the only thing that is field dependent.
        \end{itemize}
        \item $a$ is a scalar HF constant and
        \begin{equation*}
            a = g\beta A
        \end{equation*}
        \begin{itemize}
            \item $A$ is measured in Gauss; it is the \emph{experimental} HF constant.
        \end{itemize}
        \item $a$ can also be anisotropic.
        \begin{itemize}
            \item Each of the values $g_x,g_y,g_z$ can have an $a$.
            \item $a$ can be 0 for some $g_i$ and not others.
        \end{itemize}
        \item $M_I$ can have many values.
        \begin{itemize}
            \item Just like with NMR, a nuclear spin with $M_I>1/2$ can give multiline patterns.
            \item Examples.
            \begin{itemize}
                \item \ce{{}^14N} has $M_I>1/2$.
                \item TMs can go up to 7/2 or 9/2, leading to very complicated multiline coupling patterns very quickly.
            \end{itemize}
        \end{itemize}
        \item $M_I$ sums the contribution from other neighboring nuclei.
        \begin{itemize}
            \item Multiple nuclei can also give multiline patterns, notably with $2nI+1$ lines.
        \end{itemize}
        \item If a spin active nucleus is not 100\% abundant, we get a superposition.
        \begin{itemize}
            \item Cool example: \ce{Mo(CN)8^3-}.
            \begin{itemize}
                \item This is a $d^1$ $S=1/2$ compound.
                \item $I=0$ is 75\% abundant, but \ce{{}^97Mo} is 25\% abundant with $I=5/2$.
                \item Thus, an amount of intensity equivalent to 1/3 of the central signal gets divided among 6 smaller neighboring peaks that are all evenly spaced apart (see notes for image).
            \end{itemize}
        \end{itemize}
    \end{enumerate}
    \item We now get move on to higher spin species.
    \begin{itemize}
        \item This is a further level of complication.
        \item However, these species may have more accessible EPR transitions.
        \item We usually have larger splitting for higher $M_s$ values.
    \end{itemize}
    \item \textbf{Kramer's theorem}: Ions with an odd number of electrons will always have as their lowest energy level at least a doublet that will split in an applied magnetic field and give rise to an EPR signal.
    \item \textbf{Kramer's doublet}: The doublet predicted by Kramer's theorem.
    \item Example: Transitions predicted to be visible for a quartet ($S=3/2$).
    \begin{figure}[H]
        \centering
        \begin{tikzpicture}[
            every node/.style={black}
        ]
            \small
            \draw (0,4) node[above]{$E$} -- (0,0) -- (4,0) node[right]{$H$ (\si{\gauss})};
    
            \footnotesize
            \draw (1,0.1) -- ++(0,-0.2) node[below]{3000};
            \draw (2,0.1) -- ++(0,-0.2) node[below]{6000};
            \draw (3,0.1) -- ++(0,-0.2) node[below]{9000};
    
            \draw [blx,ultra thick]
                (0,3) node[left]{$M_s=+\frac{3}{2}$} -- ++(0.5,0)
                (0,1) node[left]{$M_s=-\frac{3}{2}$} -- ++(0.5,0)
            ;
            \draw [blx,semithick,name path={4}] (0.5,3) -- ++(3.3,1);
            \draw [blx,semithick,name path={3}] (0.5,3) -- ++(3.3,-1);
            \draw [blx,semithick,name path={2}] (0.5,1) -- ++(3.3,0.7);
            \draw [blx,semithick,name path={1}] (0.5,1) -- ++(3.3,-0.7);
    
            \path [name path={2000}] ({2/3},0) -- ++(0,4);
            \draw [rex,thick,name intersections={of=2000 and 3,by=a},name intersections={of=2000 and 4,by=b}] (a) -- (b);
    
            \path [name path={3000}] (1,0) -- ++(0,4);
            \draw [rex,thick,name intersections={of=3000 and 1,by=a},name intersections={of=3000 and 2,by=b}] (a) -- (b);
    
            \path [name path={9000}] (3,0) -- ++(0,4);
            \draw [rex,thick,name intersections={of=9000 and 2,by=a},name intersections={of=9000 and 3,by=b}] (a) -- (b);
        \end{tikzpicture}
        \caption{Quartet EPR signals.}
        \label{fig:EPRquartet}
    \end{figure}
    \begin{itemize}
        \item Something lower around \SI{2000}{\gauss}, higher around \SI{3000}{\gauss} and, if our spectrometer is powerful enough, something around \SI{9000}{\gauss}.
        \item Recall $\Delta M_s=\pm 1$.
    \end{itemize}
    \item Example: Transitions predicted to be visible for a triplet ($S=1$).
    \begin{itemize}
        \item Triplets are very difficult to see because the lower state takes a long time to drop low enough to see the ground state.
    \end{itemize}
    \item Treating spin-orbit coupling.
    \begin{itemize}
        \item We have that $\vec{\mu}=g\beta\vec{J}$.
        \item It follows that
        \begin{equation*}
            g = \frac{J(J+1)\cdot S(S+1)-L(L+1)}{2J(J+1)}+1
        \end{equation*}
        \item What people tend to do is look at the spectrum.
    \end{itemize}
    \item That's it for higher spin species.
    \item A few practical points on EPR spectroscopy.
    \begin{enumerate}
        \item Use a microwave cavity, a fixed frequency, and vary the field.
        \item Avoid solvents that strongly absorb microwaves (e.g., water, other polar solvents). This can sometimes be ok if the sample is frozen.
        \item Usually quote $g$-values.
        \begin{itemize}
            \item Do we report values as $H$ or $g$??
        \end{itemize}
        \item You can double integrate vs. a standard to quantity $\pm 20\%$.
    \end{enumerate}
    \item We now move on to advanced EPR techniques.
    \begin{itemize}
        \item These are analogous to some of the fancy things you can do with NMR.
        \item Essentially, they are alternate ways of getting data out of an EPR machine besides straight-up absorption spectra.
    \end{itemize}
    \item In particular, we will focus on \textbf{ENDOR} and \textbf{ESEEM} spectroscopy.
    \item \textbf{Electron spin echo envelope modulation} (spectroscopy). \emph{Also known as} \textbf{ESEEM}.
    \item \textbf{Electron nuclear double resonance} (spectroscopy). \emph{Also known as} \textbf{ENDOR}.
    \item Background.
    \begin{itemize}
        \item Most EPR spectroscopy compared to NMR is pretty rudimentary.
        \begin{itemize}
            \item In NMR, every carbon spectrum you take is proton-decoupled (we'll talk about this later).
            \item Radiofrequency is really nice to control.
        \end{itemize}
        \item However, these are both 2D experiments between EPR and NMR.
        \begin{itemize}
            \item They allow us to see spin-active nuclei that are close to the paramagnetic center.
            \item 2D EPR can give you a ton of information about paramagnetic compounds; the most useful are those that couple electron transitions with nuclear transitions.
        \end{itemize}
        \item These are super powerful techniques.
    \end{itemize}
    \item ENDOR spectroscopy.
    \begin{figure}[h!]
        \centering
        \begin{tikzpicture}[
            every node/.style={black}
        ]
            \small
            \draw (0,3) node[above]{$E$} -- (0,0) -- (5,0) node[right]{$H$};
    
            \footnotesize
            \draw (1.5,0.1) -- ++(0,-0.2) node[below]{EPR};
            \draw (2.75,0.1) -- ++(0,-0.2) node[below]{NMR};
            \draw (4,0.1) -- ++(0,-0.2) node[below]{HF};
    
            \draw [blx,ultra thick]
                (0,1.5) -- ++(0.5,0)
                (1.25,2.3) -- ++(0.5,0) node[above left]{$+\frac{1}{2}g\beta H$}
                (1.25,0.7) -- ++(0.5,0) node[below left]{$-\frac{1}{2}g\beta H$}
                (2.5,2.6) -- ++(0.5,0)
                (2.5,2) -- ++(0.5,0)
                (2.5,1) -- ++(0.5,0)
                (2.5,0.4) -- ++(0.5,0)
                (3.75,2.5) -- ++(0.9,0)
                (3.75,2.1) -- ++(0.9,0)
                (3.75,1.1) -- ++(0.9,0)
                (3.75,0.3) -- ++(0.9,0)
            ;
            \draw [blx,semithick]
                (0.5,1.5) -- (1.25,2.3)
                (0.5,1.5) -- (1.25,0.7)
                (1.75,2.3) -- (2.5,2.6)
                (1.75,2.3) -- (2.5,2)
                (1.75,0.7) -- (2.5,1)
                (1.75,0.7) -- (2.5,0.4)
                (3,2.6) -- (3.75,2.5)
                (3,2) -- (3.75,2.1)
                (3,1) -- (3.75,1.1)
                (3,0.4) -- (3.75,0.3)
            ;
            \draw [rex,thick,-stealth,shorten <=1pt,shorten >=1pt] (3.9,0.3) -- (3.9,2.1);
            \draw [rex,thick,-stealth,shorten <=1pt,shorten >=1pt] (4.1,1.1) -- (4.1,2.5);
            \draw [orx,thick,-stealth,shorten <=1pt,shorten >=1pt] (4.1,0.3) -- (4.1,1.1);
            \draw [orx,thick,-stealth,shorten <=1pt,shorten >=1pt] (3.9,2.1) -- (3.9,2.5);
            \draw [ylx,thick,-stealth,shorten <=1pt,shorten >=1pt] (4.3,1.1) -- (4.3,2.1);
            \draw [ylx,thick,-stealth,shorten <=1pt,shorten >=1pt] (4.5,0.3) -- (4.5,2.5);
    
            \draw [rex,ultra thick] (5,1.1) -- ++(0.5,0) node[right]{EPR};
            \draw [orx,ultra thick] (5,0.8) -- ++(0.5,0) node[right]{NMR};
            \draw [ylx,ultra thick] (5,0.5) -- ++(0.5,0) node[right]{ENDOR};
        \end{tikzpicture}
        \caption{ENDOR splitting.}
        \label{fig:ENDORsplitting}
    \end{figure}
    \begin{itemize}
        \item The quantum-mechanical basis for ENDOR involves the following Hamiltonian.
        \begin{equation*}
            \hat{H} = \underbrace{\vphantom{g_N}g\beta\hat{H}\cdot\hat{S}}_\text{EPR}-\underbrace{g_N\beta_N\hat{H}\cdot\hat{I}}_\text{NMR}+\underbrace{\vphantom{g_N}ha\hat{I}\cdot\hat{S}}_\text{HF}
        \end{equation*}
        \item Three levels of splitting: EPR, NMR, and HF.
        \begin{itemize}
            \item The last one doesn't actually induce splitting; it's just coupling between the first two that affects energy levels.
        \end{itemize}
        \item EPR has $\Delta M_s=\pm 1$, $\Delta M_I=0$.
        \item Here, we turn on semi-forbidden transitions, i.e., those with $\Delta M_s=\pm 1$ and $\Delta M_I=\pm 1$.
        \item $1\to 3$ and $2\to 4$ are EPR allowed.
        \item $1\to 2$ and $3\to 4$ are NMR allowed.
        \item ENDOR transitions (which we're turning on in this experiment) are $2\to 3$ and $1\to 4$.
        \item ENDOR spectra tend to look like look like a big messy hunk corresponding to the nucleus, and then distinct ones further out which correspond to the NMR transitions.
    \end{itemize}
    \item Possibly a bit more ENDOR on Friday.
    \item We now move on to ESEEM.
    \begin{itemize}
        \item Do we need to know this??
    \end{itemize}
\end{itemize}



\section{X-ray Absorption Spectroscopy}
\begin{itemize}
    \item \marginnote{2/16:}The pic that Anderson drew wasn't great; you can also find bootleg copies of \textcite{bib:Drago}.
    \item Patrick (or Pat): Guest lecturer on XAS, which he used a lot in his PhD.
    \begin{itemize}
        \item He's currently a post-doc in Anderson's group.
        \item Studied a lot of high-valent iron complexes.
    \end{itemize}
    \item We now begin the lecture.
    \item \textbf{X-ray absorption spectroscopy}: A range of techniques which are all based on the absorption of X-rays by a nucleus of choice. \emph{Also known as} \textbf{XAS}.
    \begin{figure}[h!]
        \centering
        \begin{tikzpicture}[
            every node/.style=black
        ]
            \small
            \draw (0,3.5) -- node[left]{Abs} (0,0) -- node[below]{$E$ (\si{\electronvolt})} (5,0);
            
            \footnotesize
            \draw [dashed] (2.25,0) -- ++(0,3.5);
            \node at (1.125,3.2) {XANES};
            \node at (3.625,3.2) {EXAFS};
    
            \draw [rex,thick] (0.2,0.4)
                to[out=0,in=-173] (1.5,0.5)
                to[out=7,in=180] (1.7,0.7)
                to[out=0,in=180] (2,0.5)
                to[out=0,in=180,looseness=0.3] node[fill=white]{Edge} (2.5,2.8)
                to[out=0,in=180,out looseness=0.3] (2.8,2.4)
                to[out=0,in=180] (3.2,2.6)
                to[out=0,in=180] (3.7,2.35)
                to[out=0,in=180] (4.3,2.4)
                to[out=0,in=180] (5,2.3)
            ;
    
            \node [above left=2mm] at (1.7,0.7) {Pre-edge}
                edge [shorten >=1mm] (1.7,0.7)
            ;
        \end{tikzpicture}
        \caption{An X-ray absorption spectrum.}
        \label{fig:XASspectrum}
    \end{figure}
    \begin{itemize}
        \item In a basic experiment, the absorption of X-rays is directly measured.
    \end{itemize}
    \item An X-ray with sufficient energy can eject an electron from a core orbital.
    \begin{itemize}
        \item The ejected electron (\textbf{photoelectron}) dissociates at the \textbf{edge energy}.
    \end{itemize}
    \item \textbf{Photoelectron}: An electron ejected from a core orbital by an X-ray.
    \item \textbf{Edge energy}: The energy needed to take the photoelectron to the infinite energy level.
    \item Different edges give different information.
    \item The $K$- and $L$-edges give the most important information.
    \begin{itemize}
        \item Pat uses the $L_\text{III}$ edge most commonly.
        \item It corresponds to a $2p\to 3d$ excitation.
    \end{itemize}
    \item \textbf{$\bm{K}$-edge}: The excitation of a core $1s$ electron from an element.
    \begin{itemize}
        \item The edge that can give you oxidation, symmetry, and structural information based on the photoelectron's environment.
        \item The $K$-edge energy roughly correlates with $Z$.
        \item Serena DeBeer does a lot on $K$-edge spectroscopy.
    \end{itemize}
    \item \textbf{$\bm{L}$-edge}: The excitation of a core $2s$ or $2p$ electron.
    \begin{itemize}
        \item The edge that can give you bonding and molecular orbital information.
        \item Nothing more on this today; look up Kyle Lancaster's work if you're curious.
    \end{itemize}
    \item Benefits of XAS.
    \begin{itemize}
        \item You don't need a crystal.
        \item Allows you to study transient intermediates, esp. in biochemistry; freeze-quench a reaction and get structural info about active states.
        \item Better local picture than XRD.
        \item Sensitive to local geometric structure of an absorber, i.e., ligation, oxidation state, spin state, symmetry.
        \item Lower X-ray dose needed than in crystallography; more control of photoreduction.
        \item XAS is \emph{element specific} and can be applied to almost any system --- biochemistry, synthetic inorganic chemistry, materials chemistry, etc.
        \item Can be measured at multiple edges.
        \item Tolerates less concentrated samples.
    \end{itemize}
    \item All XAS uses the same fundamental principle, but there are many detection methods to get a lot of information.
    \begin{itemize}
        \item Beamline scientists love to geek out about this.
        \item Transition mode vs. fluorescence mode.
    \end{itemize}
    \item \textbf{Transition mode}: Measure the difference between incident and final intensity.
    \begin{itemize}
        \item Better for solid samples than frozen solutions, esp. close-packed ones.
    \end{itemize}
    \item Frozen solutions with an X-ray-transparent window: Collect in fluorescence mode.
    \item To freeze: Submerge in a liquid nitrogen or isopropanol bath.
    \item K-Edge spectra.
    \begin{itemize}
        \item Two main regions: XANES (X-ray absorption near edge structure) which includes the pre-edge (symmetry information) and the $K$-edge (oxidation state info).
        \item The $K$-edge is the energy at which you actually eject the electron into the continuum.
        \begin{itemize}
            \item This energy will change with oxidation state: As you probe something that's more oxidized, you know that its electrons will be held more tightly due to decreased repulsions, so the $K$-edge will increase.
        \end{itemize}
        \item The pre-edge consists of a $1s\to 3d$ transition.
        \begin{itemize}
            \item While the edge energy is the energy required to eject an electron into the continuum, the pre-edge features correspond to excitation into valence orbitals.
        \end{itemize}
    \end{itemize}
    \item Pre-edge.
    \begin{itemize}
        \item In $K$-edge XAS, the pre-edge feature is primarily the $1s\to 3d$ transition (electron-dipole forbidden, electric-quadrupole allowed). Thus, as symmetry decreases and we get more coupling, this increases.
        \item The pre-edge area is indicative of deviation from centrosymmetry and $3d$/$4p$ mixing.
        \begin{itemize}
            \item High symmetry molecules will give $d$-$p$ mixing/hybridization and thus have more intense pre-edge features.
        \end{itemize}
        \item Selection rule: $\Delta e=\pm 1$, so technically $s\to p$ is allowed and $s\to d$ is not.
    \end{itemize}
    \item Example of XANES: \ce{Fe^V} vs. \ce{Fe^{VI}}.
    \begin{itemize}
        \item The pre-edge tends to get over-interpreted.
        \item A pure metal foil is often used as a standard; there is some debate about what a pure-metal $K$-edge is, though. Make sure you know what the standard is when you read literature!
        \item $K$-edge XANES may be the best technique we have to measure oxidation state, even though oxidation state doesn't really exist. Perhaps better for seeing that an electron was removed; much better in a comparative/relative sense than in an absolute sense.
    \end{itemize}
    \item Pre-edge area is sensitive to the oxidation state.
    \begin{itemize}
        \item This is because we often lose symmetry as we go to higher oxidation states.
        \item Even just shortening the axial ligand in a square pyramidal complex can have a drastic effect.
        \item Less symmetry $\Rightarrow$ greater peak area.
        \item You measure the area with any peak-fitting program (e.g., Fit It).
        \begin{itemize}
            \item There's a paper by Ed Solomon in the 1990s that describes fitting in greater detail; a lot of the best practices are included.
            \item You fit with a pseudo-Gaussian that's 50\% Gaussian and 50\% Voight. Sometimes it's hard to distinguish between two close peaks because there can be substantial overlap.
        \end{itemize}
        \item Metal-ligand bonding can turn on more pre-edge area; metal-ligand interactions shouldn't be too important; multiply bonded ligands with shorter oxidation states help more.
    \end{itemize}
    \item Mossbauer in theory predicts bond lengths. You get a much better correlation with bond states when...
    \item Case study: Determining $\Delta$ from the pre-edge region.
    \begin{itemize}
        \item High-spin \ce{Fe^{III}} should give two distinct $1s\to 3d$ transitions; the splitting between then is $\Delta$.
    \end{itemize}
    \item If you have a metal in different oxidation states, XANES is probably what you want.
    \begin{itemize}
        \item We use a metal standard because it's somewhat stable and...
        \item When you use foil, that's your $x$-axis calibration standard.
        \item Some compounds don't have a pre-edge.
        \item To fit our pre-edge, fit...
    \end{itemize}
    \item People in the XAS community are usually willing to discuss, share, and argue about their XAS fitting.
    \begin{itemize}
        \item Parameters varied: Computer program, function or algorithm, procedure.
    \end{itemize}
    \item Beamline scientists hate giving tutorials on how to do something or how to fix something, but they like looking at data and science.
    \item A short EXAFS primer.
    \begin{itemize}
        \item Pat's favorite part.
        \item Includes the wavy features after the edge.
        \item Mechanism: Ionizing radiation hits an absorber atom, ejecting the $1s$ electron which leaves the \textbf{absorber} atom. Then the surrounding atoms send the waves back, leading to constructive and destructive interference.
        \item You can also get multiple scattering (photoelectron may be scattered by more than one atom before returning to the center).
    \end{itemize}
    \item Equations and quantities relevant to EXAFS.
    \begin{itemize}
        \item Let $R_j$ be the distance between the absorber and back scatterer $j$.
        \item Recall that $I=I_0\e[-\mu t]$, where $I_0$ is the incident irradiation, $\mu$ is the linear absorption coefficient, and $t$ is the thickness.
        \item The signal can be modeled with the \textbf{EXAFS equation}.
        \item We have that $\chi=(\mu-\mu_s)/\mu_0$, where $\mu_s$ is with a smooth background and $\mu_0$ normalizes for free atoms.
    \end{itemize}
    \item How we get information from EXAFS: Use the EXAFS equation.
    \begin{equation*}
        \chi(k) = \sum_i\frac{(N_iS_0^2)F_i(k)}{kR_i^2}\cdot\sin(2kR_i+\delta_i(k))\e[-2\sigma_i^2k^2]\e[-2R_i/\lambda(k)]
    \end{equation*}
    \begin{itemize}
        \item The XAS spectrum is the sum of multiple sines.
        \item The sine wave overlies the fluorescence decay, which can also be important.
        \item $\chi(k)$ depends on a number of things, but for a shell made of the same scattering atoms, the variables we control are $N_i$ --- the number of equivalent scatterers.
        \item $R_i$ is the distance between your absorber and your scatterer.
        \item $f_j$ is the electron backscattering amplitude of the scatterer.
        \begin{itemize}
            \item It depends on $Z$.
        \end{itemize}
        \item $\sigma_i$ is the root mean square variation in $R_j$.
        \item $\delta_i$ is the scatterer phase shift from the theory or model.
        \item Together $\sigma_i$ and $\delta_i$ make up the \textbf{Debye-Waller factor}, which accounts for disorder.
        \item $k$ is the photoelectron vector.
        \begin{itemize}
            \item It is given by
            \begin{equation*}
                k = \sqrt{\frac{2m_e}{\hbar^2}(E-E_0)}
            \end{equation*}
            where $E_0$ is the threshold energy for electron ejection.
        \end{itemize}
        \item To accurately describe these variables, a lot of information is needed about the system...
        \item We $k$-weight our spectrum to account for differences between heavy and light elements.
        \item We FT our spectrum to transform it to space data.
    \end{itemize}
    \item \textbf{Debye-Waller factor}: Equivalent to thermal ellipsoids in crystallography. If it's negative, it's fake; if it's too big, it's fake. \emph{Given by}
    \begin{equation*}
        \e[-2\sigma_i^2k^2]
    \end{equation*}
    \item You get transitions in the edge in some specific compounds.
    \item Principles to guide data analysis.
    \begin{itemize}
        \item You can fit anything if you try hard enough. Use your chemical intuition. Does a fit make sense?
        \item Error in EXAFS is typically $\pm\SI{0.02}{\angstrom}$, but the shell resolution is usually much larger (equal to $\pi/(2\Delta K))$. Don't give things within a shell-resolution of each other!
        \item EXAFS is very much a "helper" spectroscopy.
        \begin{itemize}
            \item XAS should be the last step of any project you perform. Just to fill in the gaps; you need chemical intuition to make sense of your data.
        \end{itemize}
    \end{itemize}
    \item Information we can get out of our data: A good fit tells us\dots
    \begin{enumerate}
        \item The number, type, and distance of scatterers.
        \begin{itemize}
            \item This can inform on the immediate coordination sphere of a metal center.
            \item Heavier atoms scatter more, and shorter distances scatter more.
            \item From XAFS, you can get the number of scatterers $\pm 1$.
        \end{itemize}
        \item You can pull out angular information if the data is really good and the geometry is right.
        \begin{itemize}
            \item Pat thinks this is fake, though.
        \end{itemize}
        \item You can do it in solution phase and with amorphous materials.
        \begin{itemize}
            \item You can also pull out EXAFS data for \SIrange{6}{8}{\angstrom} in certain solids.
        \end{itemize}
    \end{enumerate}
    \item Case study: Oxygen activation at a diiron site.
    \begin{itemize}
        \item Can't do Mossbauer with iron, so even if it's not giving you approximate...
        \item Typically, you want to aim for $R(\si{\angstrom})<6$.
        \item Both irons will be oxidized...
        \item Data can be decomposed with a linear combination analysis.
    \end{itemize}
    \item Data fitting is not always so easy.
    \begin{itemize}
        \item M\"{o}ssbauer...
        \item This is the decay of 3.
        \item Thermal decay: The things we don't want to see is...
        \item Took 11 attempts to fit (that's a lot for Pat). 
    \end{itemize}
    \item Big picture for the diiron system.
    \begin{itemize}
        \item Getting more and more rigid.
    \end{itemize}
    \item Not a lot of people use this technique, so a lot of labs publish bad data, but reach out to them if you're confused! They'll usually be willing to work with you.
    \item Ask for slides!
    \item Review \textbf{XAFS for Everyone}!
    \item Return to ENDOR spectrsocopy.
    \begin{itemize}
        \item The energy level diagram we want is the one in the picture.
    \end{itemize}
    \item Experimental considerations.
    \begin{itemize}
        \item You need a high energy, intense X-ray beam (typically a synchrotron).
        \begin{itemize}
            \item The closest synchrotron to us is the APS at Argonne.
            \item We can also run at the SLAC at Stanford, and there are others elsewhere, too.
        \end{itemize}
        \item Data workup is involved and needs to be done carefully.
        \begin{itemize}
            \item The relevant software suite is called Demeter and published by Bruce Ravel; it is all free.
        \end{itemize}
    \end{itemize}
    \item A note on the data.
    \begin{itemize}
        \item After treatment, the data is typically plotted as a Fourier transform.
        \item Note: The "distance" $R$ in an FT plot is often not true distance; true distance is usually $R$ plus approximately \SI{0.4}{\angstrom}.
    \end{itemize}
    \item Next Tuesday: Mossbauer.
\end{itemize}




\end{document}