\documentclass[../psets.tex]{subfiles}

\pagestyle{main}
\renewcommand{\leftmark}{Problem Set \thesection}
\stepcounter{section}

\begin{document}




\section{Magnetism and More Spectroscopy}
\begin{enumerate}
    \item \marginnote{3/3:}You propose that you have made a terminal amide (\ce{NH2}) complex. You observe two stretching frequencies in the IR spectrum at \SI{2400}{\per\centi\meter} and \SI{2330}{\per\centi\meter}.
    \begin{enumerate}
        \item Do you have one or two complexes? Rationalize this based on symmetry.
        \item If you deuterate this complex, where should the \ce{N-D} stretches come?
    \end{enumerate}
    \item Copper (II) acetate is a dimer and the two \ce{Cu} atoms strongly interact. The EPR spectrum consists of seven lines with intensities of $1:2:3:4:3:2:1$. Copper nuclei have $I=3/2$, and copper acetate consists of a ground state singlet with an accessible triplet excited state. Explain the number and relative intensity of the observed signals.
    \item The intensities and frequencies (vs. \ce{H3PO4}) of resonances in the \SI{121.4}{\mega\hertz} \ce{{}^31P}\{\ce{{}^1H}\} NMR spectrum of \ce{Pt(PPh3)\{\eta^3-N(CH2CH2PPh2)3\}} are listed below. Determine the chemical shifts (i.e., $\delta$ values) and coupling constants, and draw the structure.
    \begin{center}
        \small
        \renewcommand{\arraystretch}{1.2}
        \begin{tabular}{SS}
            \textbf{Frequency} & \textbf{Intensity}\\
            \hline
            -3393.4 & 7.8\\
            -3313.5 & 7.7\\
            -1484.8 & 31.1\\
            -1404.9 & 31.3\\
            -833.3 & 0.7\\
            -753.5 & 1.9\\
            -673.4 & 2.0\\
            -593.6 & 0.6\\
            263.7 & 7.9\\
            343.7 & 7.7\\
            1385.5 & 2.6\\
            1465.6 & 7.8\\
            1545.8 & 7.6\\
            1625.6 & 2.7\\
            3604.5 & 0.7\\
            3684.6 & 1.9\\
            3764.4 & 1.9\\
            3844.7 & 0.6\\
        \end{tabular}
    \end{center}
    \item Predict the spin state, $\mu_\text{eff}$, and $\chi T$ values for the following ions in the indicated geometry.
    \begin{enumerate}
        \item Tetrahedral \ce{Mn(II)}.
        \item Octahedral \ce{Ir(III)}.
        \item Octahedral \ce{Ru(III)}.
        \item Square planar \ce{Co(I)}.
        \item Square planar \ce{Pt(II)}.
        \item Octahedral \ce{Ni(II)}.
        \item Tetrahedral \ce{Cr(0)}.
    \end{enumerate}
    \item Predict whether the following high-spin ions should have an isotropic EPR $g$-value greater than, equal, or less than 2.
    \begin{enumerate}
        \item \ce{Cu(II)}.
        \item \ce{Cr(V)}.
        \item \ce{Mn(II)}.
        \item \ce{Fe(III)}.
        \item \ce{Co(II)}.
    \end{enumerate}
    \item If you were handed two vials, one with ferrocene and one with ferrocenium, how could you use XAS and Mossbauer to determine which was which?
    \item 
    \begin{enumerate}
        \item Name 3 pieces of structural information you can get from a $K$-edge EXAFS spectrum on an unknown TM complex with at least one phosphine ligand.
        \item A Mossbauer spectrum of an \ce{FeL5} complex shows two signals with differing isomer shifts and quadrupole splittings. Offer an explanation for this.
    \end{enumerate}
    \item Draw a clearly labeled diagram representing the expected EPR spectrum of an aqueous \ce{{}^63Cu^2+} ion ($I=3/2$) at room temperature. Clearly indicate the position of the isotropic $g$-value.
    \item The EPR spectrum of an axially symmetric \ce{Cu^2+} complex in a frozen solution that is 100\% \ce{{}^63Cu} consists of three well-resolved low field $g$-parallel components, a fourth component of the multiplet that overlaps the $g$-perpendicular signal, and some higher field lines. The minima between the four largely resolved components appear at \SI{2720}{\gauss}, \SI{2810}{\gauss}, and \SI{2900}{\gauss}. Given this information and a spectrometer frequency of \SI{9.12}{\giga\hertz}\dots
    \begin{enumerate}
        \item Compute the value of $g_\parallel$;
        \item Compute the hyperfine coupling constant.
    \end{enumerate}
    \item The structure of the active site of carbon monoxide dehydrogenase was determined from XAS data to be
    \begin{center}
        \footnotesize
        \chemfig{Mo(-[:30]S-[:-30]Cu-[:30]S-[:-30])(=[2]O)(-[:150]@{S1}S)(-[:-150]@{S2}S)(-[:-30]OH)}
        \chemmove{
            \draw [-,shorten <=3pt,shorten >=3pt] (S1) to[bend right=40] (S2);
        }
    \end{center}
    \begin{enumerate}
        \item Indicate how you could identify the heavy atoms with XANES or EXAFS data, the oxidation state of these atoms, and the coordination environment.
        \item Your results tell you that the oxidation states of \ce{Mo} and \ce{Cu} are $6+$ and $1+$, respectively. How could you determine the protonation state of the hydroxide (\ce{OH} or \ce{OH2}) with ENDOR spectroscopy?
    \end{enumerate}
    \item For the following fragments, predict the $\chi T$ and $\mu_\text{eff}$ values for anti-ferromagnetic, ferromagnetic, and uncoupled scenarios. Using the indicated room temperature $\chi T$ or $\mu_\text{eff}$ values, predict the coupling in these fragments. Assume no contributions from spin-orbit coupling.
    \begin{enumerate}
        \item \ce{Cu^{II}-X-Cu^{II}}, $\chi T=\SI{0.4}{\centi\meter\cubed\kelvin\per\mole}$.
        \item \ce{Ni^{II}-X-Cr^{II}}, $\chi T=\SI{4.4}{\centi\meter\cubed\kelvin\per\mole}$.
        \item \ce{Fe^{III}-X-Fe^{III}}, $\mu_\text{eff}=8.4\,\mB$.
        \item \ce{Fe^{II}-X-Fe^{III}}, $\chi T=\SI{11}{\centi\meter\cubed\kelvin\per\mole}$.
    \end{enumerate}
    \item You synthesize a series of new metal oxides with each metal center in an $O_h$ coordination environment. The two metals alternate in an edge-sharing \ce{AB}. You note that when $\ce{A}=\ce{Ti(III)}$ and $\ce{B}=\ce{Cu(II)}$, the material exhibits strong ferromagnetic exchange. However, when $\ce{A}=\ce{Fe(III)}$, a strong antiferromagnetic exchange is observed. Rationalize these two observations.
\end{enumerate}




\end{document}