\documentclass[../psets.tex]{subfiles}

\pagestyle{main}
\renewcommand{\leftmark}{Problem Set \thesection}

\begin{document}




\section{Spectroscopy}
\begin{enumerate}
    \item \marginnote{1/31:}Does the wavelength of the characteristic radiation caused by bombarding of the anode material with electrons depend on voltage? Please, explain.
    \item Please, explain which effects take place in the X-ray tube upon increasing the current and voltage.
    \item Determine the maximum wavelength of the X-ray emission caused by electrons traveling with \SI{35}{\kilo\electronvolt} of energy.
    \item Which material will work better for X-ray absorption: \ce{Fe} or \ce{Pb}? Note: \ce{Fe} has atomic number and atomic weight equal to 26 and 55.8, respectively; \ce{Pb} has atomic number and atomic weight equal to 82 and 207.2, respectively. Please, explain.
    \item List three types of interactions of X-rays with matter.
    \item Define the Bravais lattice for graphene.
    \item \ce{Pt} has a structure where \ce{Pt} atoms occupy the positions at the vertices of the cube and in the middle of the facets. Please, draw the Bravais lattice for \ce{Pt}. How many atoms are in the basis for the lattice? If the \ce{Pt} atoms at the facets of the cube will be replaced with something else (e.g., \ce{Au}), will it affect the Bravais lattice?
    \item What are the types of Bravais lattices?
    \item \ce{Pt} was alloyed with \ce{Co} and a \ce{PtCo} alloy was formed. Please, analyze the peak positions measured for the \ce{Pt} and \ce{PtCo} alloy, and make a conclusion about the trend of peak positions in the XRD diffractogram of \ce{CoPt3} if it has the same lattice.
    \begin{center}
        \small
        \renewcommand{\arraystretch}{1.2}
        \begin{tabular}{|l|l|l|l|}
            \hline
             & \ce{Pt} & \ce{PtCo} & \ce{CoPt3}\\
            \hline
            $(111)$ & \num{39.036} & \num{40.252} & \\ \hline
            $(200)$ & \num{45.876} & \num{46.425} & \\ \hline
            $(220)$ & \num{66.905} & \num{68.117} & \\ \hline
            $(311)$ & \num{80.624} & \num{81.662} & \\
            \hline
        \end{tabular}
    \end{center}
    \item The shell of an unknown material was grown around \ce{Pt} nanoparticles. As a result, the peak positions of \ce{Pt} were shifted by $\sim 1.5\%$ toward the higher $2\theta$. Please, calculate the strain.
    \item Please, think about your project and come up with an experiment that involves X-ray diffraction. If this method is not relevant to the scope of your research, explain your reasoning.
\end{enumerate}




\end{document}